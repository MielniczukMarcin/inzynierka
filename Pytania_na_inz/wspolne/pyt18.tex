\section{Możliwości i ograniczenia transakcji w relacyjnych bazach danych}

\subsection{Wprowadzenie}
Transakcja w relacyjnej bazie danych to zbiór operacji wykonywanych jako jedna, niepodzielna jednostka. Każda transakcja musi spełniać zasady ACID (Atomicity, Consistency, Isolation, Durability), aby zapewnić spójność i niezawodność systemu bazodanowego.

\subsection{Możliwości transakcji w relacyjnych bazach danych}

\subsubsection{1. Spójność danych (Consistency)}
Transakcje zapewniają, że baza danych pozostaje w stanie spójnym przed i po wykonaniu transakcji. Jeśli operacja narusza integralność danych, system cofa zmiany.

\textbf{Przykład:}  
Jeśli przelewamy 100 zł z konta A na konto B, suma środków na obu kontach musi pozostać taka sama.

\begin{verbatim}
BEGIN TRANSACTION;
UPDATE Konto SET saldo = saldo - 100 WHERE id = 1;
UPDATE Konto SET saldo = saldo + 100 WHERE id = 2;
COMMIT;
\end{verbatim}

\subsubsection{2. Odporność na błędy (Atomicity)}
Transakcje są niepodzielne – jeśli któraś operacja nie powiedzie się, cała transakcja zostaje anulowana (\textit{rollback}).

\textbf{Przykład:}  
Jeśli nastąpi awaria systemu po pierwszej operacji, ale przed drugą, transakcja zostanie wycofana, aby uniknąć niespójności.

\begin{verbatim}
BEGIN TRANSACTION;
UPDATE Konto SET saldo = saldo - 100 WHERE id = 1;
IF ERROR THEN ROLLBACK;
UPDATE Konto SET saldo = saldo + 100 WHERE id = 2;
COMMIT;
\end{verbatim}

\subsubsection{3. Izolacja transakcji (Isolation)}
Zapewnia, że jednoczesne transakcje nie wpływają na siebie nawzajem. System może używać różnych poziomów izolacji:
\begin{itemize}
    \item \textbf{Read Uncommitted} – transakcje mogą odczytywać dane niezatwierdzone przez inne transakcje.
    \item \textbf{Read Committed} – transakcje odczytują tylko zatwierdzone zmiany.
    \item \textbf{Repeatable Read} – transakcja widzi te same dane przy każdym odczycie.
    \item \textbf{Serializable} – najwyższy poziom izolacji, blokuje równoczesne transakcje.
\end{itemize}

\textbf{Przykład:}  
Ustawienie izolacji transakcji w SQL Server:
\begin{verbatim}
SET TRANSACTION ISOLATION LEVEL SERIALIZABLE;
BEGIN TRANSACTION;
SELECT * FROM Konto WHERE id = 1;
COMMIT;
\end{verbatim}

\subsubsection{4. Trwałość (Durability)}
Po zatwierdzeniu transakcji (\texttt{COMMIT}), zmiany są trwale zapisane w bazie, nawet jeśli nastąpi awaria systemu.

\textbf{Przykład:}  
Po zatwierdzeniu przelewu, saldo konta jest zapisane na dysku i nie można go cofnąć w przypadku awarii.

\subsection{Ograniczenia transakcji w relacyjnych bazach danych}

\subsubsection{1. Problemy współbieżności}
Równoczesne transakcje mogą powodować konflikty:
\begin{itemize}
    \item \textbf{Dirty Read} – transakcja odczytuje dane, które mogą zostać wycofane.
    \item \textbf{Non-repeatable Read} – dane mogą zmieniać się między odczytami w jednej transakcji.
    \item \textbf{Phantom Read} – nowe wiersze mogą pojawić się między zapytaniami.
\end{itemize}

\subsubsection{2. Narzut wydajnościowy}
Wyższe poziomy izolacji (np. Serializable) mogą prowadzić do blokowania transakcji, co spowalnia działanie systemu.

\subsubsection{3. Problemy z długimi transakcjami}
Długotrwałe transakcje mogą blokować inne operacje i prowadzić do zatorów w systemie.

\subsubsection{4. Możliwość zakleszczeń (Deadlocks)}
Jeśli dwie transakcje blokują te same zasoby i czekają na siebie nawzajem, może dojść do zakleszczenia.

\textbf{Przykład:}  
Transakcja A blokuje tabelę X, a transakcja B blokuje tabelę Y – jeśli A próbuje uzyskać dostęp do Y, a B do X, powstaje zakleszczenie.

\subsubsection{5. Brak wsparcia dla rozproszonych transakcji}
Niektóre systemy bazodanowe mają ograniczone wsparcie dla transakcji obejmujących wiele baz danych.

\subsection{Podsumowanie}
\begin{itemize}
    \item Transakcje zapewniają spójność, atomowość, izolację i trwałość zmian w bazie.
    \item Mechanizmy izolacji chronią przed błędami współbieżności.
    \item Ograniczenia obejmują problemy wydajnościowe, zakleszczenia i konflikty transakcji.
    \item Odpowiednie zarządzanie poziomami izolacji pozwala na optymalizację wydajności systemu.
\end{itemize}
