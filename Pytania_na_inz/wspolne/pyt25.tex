\section{Podstawowe rodzaje licencji na oprogramowanie komputerowe w kontekście etycznej strony przestrzegania praw autorskich}

\subsection{Wprowadzenie}
Licencje na oprogramowanie określają warunki, na jakich użytkownicy mogą korzystać z programów komputerowych. Przestrzeganie praw autorskich i licencyjnych jest istotnym aspektem etyki w informatyce, ponieważ wpływa na ochronę twórczości programistów i uczciwe korzystanie z oprogramowania.

\subsection{Podstawowe rodzaje licencji}

\subsubsection{1. Licencje własnościowe (proprietary)}
Licencje własnościowe przyznają użytkownikowi ograniczone prawa do korzystania z oprogramowania, a kod źródłowy pozostaje zamknięty.

\textbf{Charakterystyka:}
\begin{itemize}
    \item Użytkownik nie ma dostępu do kodu źródłowego.
    \item Zabronione jest modyfikowanie i rozpowszechnianie oprogramowania.
    \item Producent oprogramowania ma pełną kontrolę nad jego dystrybucją i rozwojem.
\end{itemize}

\textbf{Przykłady:}
\begin{itemize}
    \item Microsoft Windows, Microsoft Office.
    \item Adobe Photoshop.
\end{itemize}

\textbf{Etyczne aspekty:}
\begin{itemize}
    \item Nielegalne kopiowanie i udostępnianie oprogramowania narusza prawa autorskie.
    \item Korzystanie z pirackiego oprogramowania jest nieetyczne i często niezgodne z prawem.
\end{itemize}

\subsubsection{2. Licencje wolnego i otwartego oprogramowania (FOSS – Free and Open Source Software)}
Licencje FOSS pozwalają użytkownikom na swobodne korzystanie, modyfikowanie i rozpowszechnianie oprogramowania.

\textbf{Charakterystyka:}
\begin{itemize}
    \item Kod źródłowy jest dostępny dla użytkowników.
    \item Możliwość modyfikowania i rozwoju oprogramowania przez społeczność.
    \item Niektóre licencje wymagają zachowania otwartości kodu w pochodnych projektach.
\end{itemize}

\textbf{Przykłady:}
\begin{itemize}
    \item Linux (licencja GPL).
    \item Mozilla Firefox (licencja MPL).
    \item LibreOffice (licencja LGPL).
\end{itemize}

\textbf{Etyczne aspekty:}
\begin{itemize}
    \item Promuje dzielenie się wiedzą i współpracę.
    \item Zachęca do uczciwego korzystania z technologii i innowacji.
\end{itemize}

\subsubsection{3. Licencje darmowe (Freeware)}
Oprogramowanie udostępniane bezpłatnie, ale zazwyczaj z ograniczeniami dotyczącymi modyfikacji lub rozpowszechniania.

\textbf{Przykłady:}
\begin{itemize}
    \item Skype.
    \item Adobe Acrobat Reader.
\end{itemize}

\textbf{Etyczne aspekty:}
\begin{itemize}
    \item Nie oznacza wolności modyfikacji – użytkownicy muszą przestrzegać warunków licencji.
    \item Korzystanie z darmowego oprogramowania zamiast pirackiego jest etycznym wyborem.
\end{itemize}

\subsubsection{4. Licencje na oprogramowanie współdzielone (Shareware)}
Użytkownik może korzystać z oprogramowania przez określony czas, po czym powinien wykupić licencję.

\textbf{Przykłady:}
\begin{itemize}
    \item WinRAR.
    \item Total Commander.
\end{itemize}

\textbf{Etyczne aspekty:}
\begin{itemize}
    \item Korzystanie z wersji próbnych jest zgodne z etyką, ale obchodzenie ograniczeń czasowych jest nieetyczne.
\end{itemize}

\subsubsection{5. Licencje publiczne (Public Domain)}
Oprogramowanie, które nie jest objęte prawami autorskimi – może być dowolnie używane, modyfikowane i rozpowszechniane.

\textbf{Przykłady:}
\begin{itemize}
    \item SQLite.
    \item Niektóre starsze wersje oprogramowania.
\end{itemize}

\textbf{Etyczne aspekty:}
\begin{itemize}
    \item Pełna swoboda użytkowania i modyfikacji.
    \item Dbanie o prawidłowe przypisywanie autorstwa jest etycznym obowiązkiem użytkowników.
\end{itemize}

\subsubsection{6. Licencje Creative Commons}
Stosowane głównie do treści cyfrowych, ale także w oprogramowaniu.

\textbf{Rodzaje:}
\begin{itemize}
    \item CC BY – dozwolone dowolne użycie pod warunkiem podania autora.
    \item CC BY-SA – wymaga udostępniania pochodnych prac na tej samej licencji.
    \item CC BY-NC – zakazuje komercyjnego wykorzystania.
\end{itemize}

\textbf{Przykłady:}
\begin{itemize}
    \item Dokumentacja projektów open-source.
    \item Materiały edukacyjne.
\end{itemize}

\subsection{Podsumowanie}
\begin{itemize}
    \item Licencje definiują zasady korzystania z oprogramowania i chronią prawa autorskie.
    \item Wybór licencji wpływa na dostępność i rozwój oprogramowania.
    \item Etyczne korzystanie z oprogramowania obejmuje przestrzeganie licencji i unikanie piractwa.
\end{itemize}
