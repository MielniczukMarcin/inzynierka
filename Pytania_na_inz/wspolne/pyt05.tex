\section{Różnice pomiędzy układami kombinacyjnymi i sekwencyjnymi oraz znaczenie takiego podziału w procesie projektowania}

\subsection{Wprowadzenie}
Układy logiczne dzielimy na:
\begin{itemize}
    \item \textbf{Układy kombinacyjne} – wyjście zależy wyłącznie od aktualnych wartości wejściowych.
    \item \textbf{Układy sekwencyjne} – wyjście zależy zarówno od aktualnych wartości wejściowych, jak i od poprzednich stanów systemu.
\end{itemize}

Podział ten jest kluczowy w projektowaniu systemów cyfrowych, ponieważ pozwala określić sposób przetwarzania informacji i sterowania.

\subsection{Układy kombinacyjne}
\subsubsection{Definicja}
Układy kombinacyjne to układy logiczne, w których wyjście zależy wyłącznie od wartości wejść w danej chwili, bez pamięci stanów poprzednich.

\subsubsection{Charakterystyka}
\begin{itemize}
    \item Brak elementów pamiętających – brak rejestrowania poprzednich wartości sygnałów.
    \item Wyjście zmienia się natychmiast po zmianie wejścia (z niewielkim opóźnieniem wynikającym z propagacji sygnału).
    \item Opis układu można wyrazić za pomocą funkcji logicznych zmiennych wejściowych.
\end{itemize}

\subsubsection{Przykłady układów kombinacyjnych}
\begin{itemize}
    \item Bramki logiczne (AND, OR, NOT, XOR).
    \item Sumatory (dodające liczby binarne).
    \item Multipleksery i demultipleksery.
    \item Enkodery i dekodery.
    \item Układy realizujące tablice prawdy.
\end{itemize}

\subsection{Układy sekwencyjne}
\subsubsection{Definicja}
Układy sekwencyjne to układy logiczne, w których wyjście zależy zarówno od aktualnych wartości wejściowych, jak i od wcześniejszego stanu układu.

\subsubsection{Charakterystyka}
\begin{itemize}
    \item Posiadają elementy pamiętające (np. przerzutniki).
    \item Reagują na zmiany wejść, ale ich stan zależy także od wcześniejszych wartości sygnałów.
    \item Do ich opisu stosuje się tabele stanów i diagramy przejść.
\end{itemize}

\subsubsection{Przykłady układów sekwencyjnych}
\begin{itemize}
    \item Liczniki (np. licznik binarny, Johnsona, pierścieniowy).
    \item Rejestry przesuwnych.
    \item Automaty skończone (np. układy sterujące).
    \item Pamięci cyfrowe (np. SRAM, DRAM).
\end{itemize}

\subsection{Różnice między układami kombinacyjnymi i sekwencyjnymi}

\begin{table}[h]
    \centering
    \renewcommand{\arraystretch}{1.3}
    \begin{tabular}{|c|c|c|}
        \hline
        \textbf{Cecha} & \textbf{Układy kombinacyjne} & \textbf{Układy sekwencyjne} \\
        \hline
        \textbf{Zależność wyjścia} & Tylko od wejść & Od wejść i poprzednich stanów \\
        \hline
        \textbf{Elementy pamięci} & Brak & Przerzutniki, rejestry \\
        \hline
        \textbf{Czas odpowiedzi} & Natychmiastowy & Wymaga taktowania \\
        \hline
        \textbf{Opis matematyczny} & Funkcje boolowskie & Tabele i diagramy stanów \\
        \hline
        \textbf{Przykłady} & Bramki, sumatory & Liczniki, rejestry \\
        \hline
    \end{tabular}
    \caption{Porównanie układów kombinacyjnych i sekwencyjnych}
\end{table}


\subsection{Znaczenie podziału w procesie projektowania}
Podział na układy kombinacyjne i sekwencyjne jest kluczowy w projektowaniu systemów cyfrowych. Pozwala on na:
\begin{itemize}
    \item \textbf{Optymalizację} – wybór odpowiedniej struktury minimalizuje zużycie zasobów sprzętowych.
    \item \textbf{Podział funkcjonalny} – rozdzielenie logiki sterującej (sekwencyjnej) i przetwarzania danych (kombinacyjnego).
    \item \textbf{Projektowanie układów synchronicznych} – układy sekwencyjne wymagają synchronizacji zegara, co jest istotne dla działania procesorów i pamięci.
    \item \textbf{Elastyczność w implementacji} – układy kombinacyjne można realizować w układach FPGA i ASIC, a układy sekwencyjne są podstawą mikroprocesorów.
\end{itemize}

\subsection{Podsumowanie}
\begin{itemize}
    \item Układy kombinacyjne realizują funkcje logiczne bez pamięci stanu, a układy sekwencyjne wykorzystują pamięć do przechowywania poprzednich stanów.
    \item Układy sekwencyjne są bardziej złożone, ale niezbędne w sterowaniu i automatyce.
    \item Podział ten jest kluczowy w projektowaniu systemów cyfrowych i pozwala efektywnie organizować architekturę układów elektronicznych.
\end{itemize}