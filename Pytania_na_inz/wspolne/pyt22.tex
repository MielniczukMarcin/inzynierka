\section{Porównanie protokołów TCP i UDP}

\subsection{Wprowadzenie}
TCP (Transmission Control Protocol) i UDP (User Datagram Protocol) to dwa główne protokoły transportowe stosowane w sieciach komputerowych. Oba służą do przesyłania danych, jednak różnią się sposobem działania i zastosowaniem.

\subsection{Charakterystyka protokołu TCP}
TCP to protokół połączeniowy, zapewniający niezawodny transfer danych.

\textbf{Cechy TCP:}
\begin{itemize}
    \item \textbf{Połączeniowy} – wymaga nawiązania sesji między nadawcą a odbiorcą.
    \item \textbf{Gwarantuje dostarczenie danych} – wykorzystuje mechanizmy retransmisji w razie utraty pakietów.
    \item \textbf{Kontrola przepływu} – dostosowuje prędkość transmisji do możliwości odbiorcy.
    \item \textbf{Podział na segmenty} – dane są dzielone na segmenty, numerowane i składane w odpowiedniej kolejności.
    \item \textbf{Wykrywanie błędów} – wykorzystuje sumy kontrolne.
\end{itemize}

\textbf{Przykłady zastosowań TCP:}
\begin{itemize}
    \item Przeglądanie stron WWW (HTTP, HTTPS).
    \item Transfer plików (FTP).
    \item Poczta elektroniczna (SMTP, IMAP, POP3).
\end{itemize}

\subsection{Charakterystyka protokołu UDP}
UDP to protokół bezpołączeniowy, który oferuje szybki, ale mniej niezawodny transfer danych.

\textbf{Cechy UDP:}
\begin{itemize}
    \item \textbf{Bezpołączeniowy} – nie wymaga ustanowienia sesji przed przesłaniem danych.
    \item \textbf{Brak gwarancji dostarczenia} – pakiety mogą ginąć lub docierać w innej kolejności.
    \item \textbf{Brak retransmisji} – utracone pakiety nie są ponownie przesyłane.
    \item \textbf{Niskie opóźnienia} – nadaje się do transmisji wymagających minimalnego czasu reakcji.
\end{itemize}

\textbf{Przykłady zastosowań UDP:}
\begin{itemize}
    \item Transmisje strumieniowe audio i wideo (VoIP, IPTV).
    \item Gry online wymagające szybkiej wymiany danych.
    \item Protokoły DNS i DHCP.
\end{itemize}

\subsection{Porównanie TCP i UDP}

\begin{table}[h]
    \centering
    \renewcommand{\arraystretch}{1.3}
    \begin{tabular}{|c|c|c|}
        \hline
        \textbf{Cecha} & \textbf{TCP} & \textbf{UDP} \\
        \hline
        \textbf{Typ protokołu} & Połączeniowy & Bezpołączeniowy \\
        \hline
        \textbf{Gwarancja dostarczenia} & Tak (retransmisja) & Nie \\
        \hline
        \textbf{Kolejność pakietów} & Zachowana & Może być losowa \\
        \hline
        \textbf{Kontrola błędów} & Tak & Minimalna \\
        \hline
        \textbf{Kontrola przepływu} & Tak & Nie \\
        \hline
        \textbf{Szybkość} & Wolniejszy (ze względu na kontrolę) & Szybszy \\
        \hline
        \textbf{Zastosowania} & HTTP, FTP, e-mail & VoIP, DNS, gry online \\
        \hline
    \end{tabular}
    \caption{Porównanie protokołów TCP i UDP}
\end{table}

\subsection{Podsumowanie}
\begin{itemize}
    \item TCP zapewnia niezawodność i kontrolę transmisji, ale kosztem wydajności.
    \item UDP jest szybszy i lepiej nadaje się do zastosowań wymagających niskich opóźnień.
    \item Wybór protokołu zależy od specyfiki aplikacji – TCP dla aplikacji wymagających niezawodności, UDP dla aplikacji czasu rzeczywistego.
\end{itemize}
