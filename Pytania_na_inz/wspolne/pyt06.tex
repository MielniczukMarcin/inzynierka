\section{Podstawowe cechy języka programowania, kwalifikujące do zaliczenia do grupy języków zorientowanych obiektowo}

\subsection{Wprowadzenie}
Programowanie obiektowe (ang. \textit{Object-Oriented Programming, OOP}) to paradygmat programowania, który organizuje kod wokół obiektów – jednostek łączących dane oraz metody operujące na tych danych. Język programowania można uznać za obiektowy, jeśli spełnia pewne kluczowe cechy i zasady.

\subsection{Podstawowe cechy języków obiektowych}
Język programowania jest klasyfikowany jako obiektowy, jeśli wspiera następujące koncepcje:

\subsubsection{1. Abstrakcja (\textit{Abstraction})}
Abstrakcja polega na ukrywaniu szczegółów implementacji i eksponowaniu tylko istotnych właściwości obiektów. Dzięki temu użytkownik korzysta z interfejsu klasy, nie znając jej wewnętrznych mechanizmów.

\textbf{Przykład w C++}:
\begin{verbatim}
class Samochod {
private:
    string marka;
public:
    void ustawMarke(string m) { marka = m; }
    string pobierzMarke() { return marka; }
};
\end{verbatim}

\subsubsection{2. Hermetyzacja (\textit{Encapsulation})}
Hermetyzacja oznacza ograniczenie dostępu do wewnętrznych danych obiektu. Osiąga się to poprzez modyfikatory dostępu (\texttt{private}, \texttt{protected}, \texttt{public}) i ukrycie zmiennych wewnątrz klasy.

\textbf{Zalety hermetyzacji:}
\begin{itemize}
    \item Ochrona przed nieautoryzowanym dostępem do danych.
    \item Zapewnienie integralności danych poprzez kontrolowany dostęp.
\end{itemize}

\subsubsection{3. Dziedziczenie (\textit{Inheritance})}
Dziedziczenie pozwala na tworzenie nowych klas na podstawie już istniejących. Klasa pochodna (\textit{subclass}) dziedziczy cechy i metody klasy bazowej (\textit{superclass}), co pozwala na ponowne użycie kodu.

\textbf{Przykład w C++}:
\begin{verbatim}
class Pojazd {
public:
    void info() { cout << "To jest pojazd"; }
};

class Samochod : public Pojazd {
};
\end{verbatim}
Tutaj \texttt{Samochod} dziedziczy metodę \texttt{info()} z klasy \texttt{Pojazd}.

\subsubsection{4. Polimorfizm (\textit{Polymorphism})}
Polimorfizm umożliwia definiowanie wielu wersji tej samej metody, działających na różnych typach danych. Wyróżnia się:
\begin{itemize}
    \item \textbf{Polimorfizm statyczny} (przeciążanie funkcji i operatorów).
    \item \textbf{Polimorfizm dynamiczny} (metody wirtualne i nadpisywanie metod).
\end{itemize}

\textbf{Przykład polimorfizmu dynamicznego w C++}:
\begin{verbatim}
class Pojazd {
public:
    virtual void dzwiek() { cout << "Ogólny dźwięk pojazdu"; }
};

class Samochod : public Pojazd {
public:
    void dzwiek() override { cout << "Dźwięk silnika samochodu"; }
};
\end{verbatim}

\subsection{Dodatkowe cechy języków obiektowych}
Oprócz czterech podstawowych filarów OOP, języki obiektowe często wspierają również inne mechanizmy:

\subsubsection{1. Klasy i obiekty}
Klasy są szablonami do tworzenia obiektów, a obiekty są instancjami klas.

\textbf{Przykład w C++}:
\begin{verbatim}
class Osoba {
public:
    string imie;
    int wiek;
};

int main() {
    Osoba o1;
    o1.imie = "Jan";
    o1.wiek = 30;
}
\end{verbatim}

\subsubsection{2. Konstruktor i destruktor}
Konstruktor to specjalna metoda wywoływana podczas tworzenia obiektu, a destruktor – podczas usuwania obiektu.

\begin{verbatim}
class Samochod {
public:
    Samochod() { cout << "Tworzę samochód"; }
    ~Samochod() { cout << "Usuwam samochód"; }
};
\end{verbatim}

\subsubsection{3. Interfejsy i klasy abstrakcyjne}
Interfejsy i klasy abstrakcyjne umożliwiają definiowanie wspólnych metod dla różnych klas.

\textbf{Przykład w C++}:
\begin{verbatim}
class Figura {
public:
    virtual void rysuj() = 0; // Klasa abstrakcyjna
};
\end{verbatim}

\subsection{Znaczenie obiektowego podejścia w projektowaniu oprogramowania}
Programowanie obiektowe ma kluczowe znaczenie w projektowaniu oprogramowania:
\begin{itemize}
    \item \textbf{Reużywalność kodu} – dzięki dziedziczeniu możliwe jest ponowne użycie już istniejącego kodu.
    \item \textbf{Łatwiejsza konserwacja} – hermetyzacja zapewnia lepszą kontrolę nad modyfikacjami kodu.
    \item \textbf{Lepsza organizacja kodu} – klasy i obiekty pozwalają na logiczne grupowanie danych i metod.
    \item \textbf{Łatwiejsza skalowalność} – programowanie obiektowe umożliwia łatwe dodawanie nowych funkcjonalności.
\end{itemize}

\subsection{Przykłady języków zorientowanych obiektowo}
\begin{itemize}
    \item \textbf{C++} – język wieloparadygmatowy, umożliwiający zarówno programowanie strukturalne, jak i obiektowe.
    \item \textbf{Java} – język w pełni obiektowy, gdzie każda klasa dziedziczy po klasie nadrzędnej \texttt{Object}.
    \item \textbf{Python} – dynamiczny język z pełnym wsparciem dla programowania obiektowego.
    \item \textbf{C\#} – język silnie związany z platformą .NET, wspierający zarówno OOP, jak i programowanie funkcyjne.
\end{itemize}

\subsection{Podsumowanie}
\begin{itemize}
    \item Języki obiektowe charakteryzują się abstrakcją, hermetyzacją, dziedziczeniem i polimorfizmem.
    \item Obiektowe podejście ułatwia organizację kodu, reużywalność i konserwację.
    \item Do najpopularniejszych języków OOP należą C++, Java, Python i C\#.
\end{itemize}
