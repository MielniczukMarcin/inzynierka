\section{Pojęcie i kategorie wynalazku}

\subsection{Definicja wynalazku}
Wynalazek to nowe i użyteczne rozwiązanie techniczne dotyczące produktu lub sposobu wytwarzania, które można zastosować w przemyśle. Wynalazki są chronione prawnie poprzez patenty, które zapewniają wyłączne prawo do korzystania z wynalazku przez określony czas.

Zgodnie z \textbf{Ustawą – Prawo własności przemysłowej} (Dz.U. 2001 nr 49 poz. 508), wynalazkiem jest rozwiązanie techniczne spełniające następujące warunki:
\begin{itemize}
    \item \textbf{Nowość} – wynalazek nie może być częścią stanu techniki.
    \item \textbf{Poziom wynalazczy} – nie może być oczywisty dla osoby znającej stan techniki.
    \item \textbf{Przemysłowa stosowalność} – wynalazek musi mieć zastosowanie w przemyśle, rolnictwie, leśnictwie lub innych dziedzinach gospodarki.
\end{itemize}

\subsection{Kategorie wynalazków}
Wynalazki można podzielić na kilka kategorii w zależności od ich charakteru i zastosowania.

\subsubsection{1. Wynalazki produktowe}
Obejmują nowe urządzenia, substancje chemiczne, materiały i kompozycje. Wynalazki produktowe mogą obejmować:
\begin{itemize}
    \item Nowe materiały (np. stop metalu, polimer).
    \item Leki i substancje farmaceutyczne.
    \item Urządzenia techniczne (np. nowy typ silnika, układu elektronicznego).
\end{itemize}

\subsubsection{2. Wynalazki procesowe}
Dotyczą nowych sposobów wytwarzania lub przetwarzania materiałów i produktów. Przykłady:
\begin{itemize}
    \item Nowa metoda syntezy chemicznej.
    \item Usprawniony proces produkcji mikroprocesorów.
    \item Technika zwiększająca efektywność energetyczną.
\end{itemize}

\subsubsection{3. Wynalazki dotyczące zastosowania}
Obejmują nowe sposoby użycia znanych substancji, urządzeń lub metod w nowym kontekście. Przykłady:
\begin{itemize}
    \item Nowe zastosowanie znanego leku w leczeniu innej choroby.
    \item Wykorzystanie istniejącej technologii w nowej branży.
\end{itemize}

\subsubsection{4. Wynalazki dotyczące oprogramowania}
Dotyczą algorytmów i metod obliczeniowych, jeśli mają zastosowanie techniczne. Przykłady:
\begin{itemize}
    \item Nowe metody kompresji danych.
    \item Algorytmy kryptograficzne.
    \item Sposób sterowania urządzeniem przez oprogramowanie.
\end{itemize}

\subsubsection{5. Wynalazki biotechnologiczne}
Obejmują odkrycia w dziedzinie biologii, medycyny i inżynierii genetycznej. Przykłady:
\begin{itemize}
    \item Modyfikacje genetyczne roślin.
    \item Nowe szczepy bakterii wykorzystywane w przemyśle farmaceutycznym.
    \item Metody inżynierii tkankowej.
\end{itemize}

\subsection{Ograniczenia w zakresie patentowania}
Nie wszystkie rozwiązania mogą zostać opatentowane. Zgodnie z przepisami prawa patentowego nie podlegają ochronie:
\begin{itemize}
    \item Odkrycia naukowe i teorie matematyczne.
    \item Metody działalności gospodarczej, finansowej i organizacyjnej.
    \item Programy komputerowe jako takie (bez zastosowania technicznego).
    \item Wynalazki sprzeczne z porządkiem publicznym lub moralnością.
\end{itemize}

\subsection{Podsumowanie}
\begin{itemize}
    \item Wynalazek to rozwiązanie techniczne spełniające kryteria nowości, poziomu wynalazczego i przemysłowej stosowalności.
    \item Można wyróżnić wynalazki produktowe, procesowe, biotechnologiczne, zastosowania oraz oprogramowania.
    \item Nie wszystkie rozwiązania mogą być opatentowane – prawo wyklucza np. odkrycia naukowe i metody organizacyjne.
\end{itemize}
