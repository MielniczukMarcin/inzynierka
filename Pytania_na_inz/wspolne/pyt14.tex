\section{Charakterystyka wybranego modelu oświetlenia oraz jego komponentów}

\subsection{Wprowadzenie}
Modele oświetlenia w grafice komputerowej symulują sposób, w jaki światło oddziałuje z obiektami, aby uzyskać realistyczny wygląd sceny. Oświetlenie wpływa na percepcję kształtu, głębi i materiałów obiektów. Istnieje wiele modeli oświetlenia, jednak jednym z najczęściej stosowanych w grafice trójwymiarowej jest \textbf{model Phonga}.

\subsection{Model oświetlenia Phonga}
Model Phonga jest empirycznym modelem oświetlenia, który opisuje sposób, w jaki światło odbija się od powierzchni. Składa się z trzech głównych komponentów:
\begin{equation}
I = I_a + I_d + I_s
\end{equation}
gdzie:
\begin{itemize}
    \item \( I_a \) – składnik oświetlenia otoczenia (ambient),
    \item \( I_d \) – składnik oświetlenia rozproszonego (diffuse),
    \item \( I_s \) – składnik oświetlenia odbitego (specular).
\end{itemize}

\subsubsection{1. Składnik oświetlenia otoczenia (\textit{Ambient Light})}
Jest to globalny składnik światła, który symuluje efekt światła odbitego od wielu powierzchni. Modeluje oświetlenie w obszarach, do których bezpośrednie światło nie dociera.

\textbf{Wzór:}
\begin{equation}
I_a = k_a I_L
\end{equation}
gdzie:
\begin{itemize}
    \item \( k_a \) – współczynnik odbicia światła otoczenia,
    \item \( I_L \) – natężenie światła otoczenia.
\end{itemize}

\subsubsection{2. Składnik oświetlenia rozproszonego (\textit{Diffuse Light})}
Modeluje rozpraszanie światła na powierzchni obiektu. Natężenie światła zależy od kąta padania światła na powierzchnię i opisuje efekt, w którym powierzchnie ustawione prostopadle do kierunku światła są jaśniejsze.

\textbf{Wzór:}
\begin{equation}
I_d = k_d I_L \max(0, \vec{N} \cdot \vec{L})
\end{equation}
gdzie:
\begin{itemize}
    \item \( k_d \) – współczynnik odbicia światła rozproszonego,
    \item \( \vec{N} \) – wektor normalny do powierzchni,
    \item \( \vec{L} \) – wektor kierunku światła.
\end{itemize}

\subsubsection{3. Składnik oświetlenia odbitego (\textit{Specular Light})}
Symuluje efekt połysku i refleksów świetlnych na powierzchniach błyszczących. Wartość ta zależy od kąta między wektorem odbicia światła a kierunkiem patrzenia.

\textbf{Wzór:}
\begin{equation}
I_s = k_s I_L \max(0, \vec{R} \cdot \vec{V})^n
\end{equation}
gdzie:
\begin{itemize}
    \item \( k_s \) – współczynnik odbicia światła zwierciadlanego,
    \item \( \vec{R} \) – wektor odbitego światła,
    \item \( \vec{V} \) – wektor kierunku obserwacji,
    \item \( n \) – współczynnik połysku (im wyższy, tym mniejsze i bardziej skoncentrowane refleksy).
\end{itemize}

\subsection{Rodzaje źródeł światła}
W modelu Phonga mogą występować różne typy źródeł światła:
\begin{itemize}
    \item \textbf{Światło punktowe} – emituje światło we wszystkich kierunkach z jednego punktu.
    \item \textbf{Światło kierunkowe} – symuluje światło pochodzące z bardzo odległego źródła (np. słońca), promienie światła są równoległe.
    \item \textbf{Światło reflektorowe} – podobne do światła punktowego, ale ograniczone do stożka.
\end{itemize}

\subsection{Zalety modelu Phonga}
\begin{itemize}
    \item Prosta i szybka implementacja w porównaniu do bardziej zaawansowanych metod.
    \item Dobrze odwzorowuje realistyczne oświetlenie, w tym połysk powierzchni.
    \item Może być stosowany w renderingu czasu rzeczywistego.
\end{itemize}

\subsection{Wady modelu Phonga}
\begin{itemize}
    \item Nie uwzględnia efektów globalnego oświetlenia, takich jak odbicia czy załamania światła.
    \item Zakłada jednolite odbicie światła dla całej powierzchni, co może powodować brak realizmu.
\end{itemize}

\subsection{Podsumowanie}
Model Phonga jest jednym z najczęściej stosowanych modeli oświetlenia w grafice komputerowej. Składa się z trzech komponentów: oświetlenia otoczenia, rozproszonego i odbitego. Mimo swoich ograniczeń, jego prostota i efektywność sprawiają, że jest szeroko stosowany w grafice 3D, szczególnie w aplikacjach działających w czasie rzeczywistym.
