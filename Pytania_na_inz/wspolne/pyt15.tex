\section{Metody zwielokrotnienia kanałów transmisyjnych oraz media transmisyjne}

\subsection{Wprowadzenie}
Zwielokrotnienie kanałów transmisyjnych to technika umożliwiająca jednoczesne przesyłanie wielu sygnałów przez jedno medium transmisyjne. Pozwala to na efektywne wykorzystanie dostępnej przepustowości oraz zwiększenie liczby użytkowników korzystających z tej samej infrastruktury sieciowej.

\subsection{Metody zwielokrotnienia kanałów transmisyjnych}

\subsubsection{1. Zwielokrotnienie częstotliwościowe (FDM – Frequency Division Multiplexing)}
FDM polega na podziale pasma częstotliwościowego medium transmisyjnego na wiele niepokrywających się zakresów, z których każdy jest wykorzystywany do przesyłania innego sygnału.

\textbf{Zastosowania:}
\begin{itemize}
    \item Radio i telewizja analogowa (np. kanały radiowe na różnych częstotliwościach).
    \item Systemy telefonii analogowej.
    \item Sieci światłowodowe (technologia WDM).
\end{itemize}

\textbf{Zalety:}
\begin{itemize}
    \item Umożliwia jednoczesną transmisję wielu sygnałów.
    \item Brak konieczności synchronizacji czasowej.
\end{itemize}

\textbf{Wady:}
\begin{itemize}
    \item Może prowadzić do zakłóceń międzykanałowych (interferencja sąsiednich pasm).
    \item Wymaga zastosowania filtrów pasmowych.
\end{itemize}

\subsubsection{2. Zwielokrotnienie czasowe (TDM – Time Division Multiplexing)}
TDM przydziela każdemu sygnałowi dostęp do kanału transmisyjnego na określony przedział czasu (slot czasowy).

\textbf{Zastosowania:}
\begin{itemize}
    \item Telefonia cyfrowa (np. systemy ISDN, E1/T1).
    \item Sieci komputerowe (np. technologia SONET/SDH).
\end{itemize}

\textbf{Zalety:}
\begin{itemize}
    \item Wysoka efektywność wykorzystania pasma.
    \item Brak interferencji między sygnałami.
\end{itemize}

\textbf{Wady:}
\begin{itemize}
    \item Wymaga precyzyjnej synchronizacji.
    \item Opóźnienia w transmisji pakietów.
\end{itemize}

\subsubsection{3. Zwielokrotnienie kodowe (CDM – Code Division Multiplexing)}
CDM polega na przypisaniu każdemu użytkownikowi unikalnego kodu, dzięki czemu wiele sygnałów może być przesyłanych równocześnie w tym samym paśmie częstotliwościowym.

\textbf{Zastosowania:}
\begin{itemize}
    \item Sieci telefonii komórkowej (np. UMTS, CDMA).
    \item Systemy satelitarne.
\end{itemize}

\textbf{Zalety:}
\begin{itemize}
    \item Wysoka odporność na zakłócenia.
    \item Możliwość dynamicznego przydzielania zasobów.
\end{itemize}

\textbf{Wady:}
\begin{itemize}
    \item Skomplikowane techniki kodowania i dekodowania.
    \item Większe zapotrzebowanie na moc obliczeniową.
\end{itemize}

\subsubsection{4. Zwielokrotnienie przestrzenne (SDM – Space Division Multiplexing)}
SDM wykorzystuje fizyczną separację kanałów transmisyjnych, np. poprzez zastosowanie wielu przewodów lub anten.

\textbf{Zastosowania:}
\begin{itemize}
    \item Światłowody wielordzeniowe.
    \item Systemy MIMO (Multiple Input Multiple Output) w sieciach 4G i 5G.
\end{itemize}

\textbf{Zalety:}
\begin{itemize}
    \item Możliwość znacznego zwiększenia przepustowości.
    \item Brak interferencji między kanałami.
\end{itemize}

\textbf{Wady:}
\begin{itemize}
    \item Wymaga większej liczby urządzeń nadawczo-odbiorczych.
\end{itemize}

\subsubsection{5. Zwielokrotnienie długości fali (WDM – Wavelength Division Multiplexing)}
WDM stosowane w światłowodach pozwala na przesyłanie wielu sygnałów optycznych o różnych długościach fali w tym samym medium transmisyjnym.

\textbf{Zastosowania:}
\begin{itemize}
    \item Sieci światłowodowe dalekiego zasięgu (np. DWDM – Dense WDM).
\end{itemize}

\textbf{Zalety:}
\begin{itemize}
    \item Ogromna przepustowość transmisji.
    \item Niski poziom zakłóceń.
\end{itemize}

\textbf{Wady:}
\begin{itemize}
    \item Wysoki koszt infrastruktury.
\end{itemize}

\subsection{Media transmisyjne}
Media transmisyjne dzielą się na przewodowe i bezprzewodowe.

\subsubsection{1. Media przewodowe}
\begin{itemize}
    \item \textbf{Przewody miedziane} – stosowane w sieciach telefonicznych i Ethernet (skrętka UTP/STP).
    \item \textbf{Światłowody} – wykorzystujące fale świetlne do transmisji, zapewniające bardzo dużą przepustowość.
    \item \textbf{Kable koncentryczne} – stosowane w telewizji kablowej i starszych sieciach Ethernet.
\end{itemize}

\subsubsection{2. Media bezprzewodowe}
\begin{itemize}
    \item \textbf{Fale radiowe} – stosowane w sieciach Wi-Fi, Bluetooth i telefonii komórkowej.
    \item \textbf{Podczerwień} – wykorzystywana w pilotach zdalnego sterowania.
    \item \textbf{Fale mikrofalowe} – stosowane w komunikacji satelitarnej i punkt-punkt.
\end{itemize}

\subsection{Podsumowanie}
\begin{itemize}
    \item Istnieje wiele metod zwielokrotnienia kanałów transmisyjnych, w tym FDM, TDM, CDM, SDM i WDM.
    \item Każda z metod ma swoje zalety i ograniczenia, zależne od zastosowania.
    \item Media transmisyjne mogą być przewodowe (miedź, światłowody) lub bezprzewodowe (fale radiowe, mikrofale).
\end{itemize}
