\section{Hierarchia pamięci, z uwzględnieniem pamięci podręcznej oraz pamięci wirtualnej}

\subsection{Wprowadzenie}
Hierarchia pamięci to struktura organizacyjna systemu komputerowego, która ma na celu optymalne zarządzanie danymi i ich dostępnością. Pamięci o wysokiej szybkości są kosztowne i mają ograniczoną pojemność, natomiast pamięci o dużej pojemności są wolniejsze. Dlatego stosuje się wielopoziomową hierarchię, w której pamięć szybsza działa jako bufor dla pamięci wolniejszej.

\subsection{Poziomy hierarchii pamięci}
Hierarchia pamięci składa się z kilku poziomów, uporządkowanych według rosnącego czasu dostępu i malejącej szybkości:

\begin{itemize}
    \item \textbf{Rejestry procesora} – najszybsza pamięć, bezpośrednio dostępna przez CPU.
    \item \textbf{Pamięć podręczna (cache)} – szybka pamięć buforowa, redukująca liczbę dostępów do pamięci RAM.
    \item \textbf{Pamięć operacyjna (RAM)} – główna pamięć komputera, przechowująca dane i kod wykonywanego programu.
    \item \textbf{Pamięć wirtualna} – mechanizm symulujący dodatkową pamięć RAM na dysku twardym.
    \item \textbf{Pamięć masowa} – dyski SSD, HDD, nośniki optyczne przechowujące dane trwałe.
    \item \textbf{Pamięć zewnętrzna} – nośniki wymienne, serwery sieciowe, chmura obliczeniowa.
\end{itemize}

\subsection{Pamięć podręczna (cache)}
Pamięć podręczna jest małą, ale bardzo szybką pamięcią umieszczoną blisko jednostki centralnej (CPU). Przechowuje często używane dane, co znacząco przyspiesza działanie systemu.

\subsubsection{Poziomy pamięci cache}
\begin{itemize}
    \item \textbf{L1} – najszybsza i najmniejsza (kilka-kilkadziesiąt KB), bezpośrednio połączona z rdzeniem procesora.
    \item \textbf{L2} – większa (setki KB – kilka MB), współdzielona przez kilka rdzeni.
    \item \textbf{L3} – najwolniejsza z cache procesora, ale największa (kilka MB – kilkadziesiąt MB), wspólna dla wszystkich rdzeni.
\end{itemize}

\subsubsection{Zasada działania pamięci podręcznej}
Dane w pamięci podręcznej są przechowywane zgodnie z zasadą lokalności:
\begin{itemize}
    \item \textbf{Lokalność czasowa} – jeśli dany blok pamięci został użyty, istnieje duże prawdopodobieństwo, że wkrótce zostanie ponownie użyty.
    \item \textbf{Lokalność przestrzenna} – jeśli odczytano pewien obszar pamięci, sąsiednie adresy prawdopodobnie będą wkrótce potrzebne.
\end{itemize}

\subsubsection{Strategie zarządzania pamięcią podręczną}
\begin{itemize}
    \item \textbf{Mapowanie bezpośrednie} – każdemu blokowi pamięci RAM odpowiada dokładnie jedna linia cache.
    \item \textbf{Mapowanie skojarzeniowe} – każdy blok pamięci może być przechowywany w dowolnym miejscu cache.
    \item \textbf{Mapowanie skojarzeniowe zestawowe} – blok pamięci może być umieszczony w jednym z kilku określonych miejsc.
\end{itemize}

\subsubsection{Wskaźnik trafień (hit rate)}
Efektywność pamięci podręcznej określa się za pomocą wskaźnika trafień:
\[
Hit\ rate = \frac{\text{Liczba trafień}}{\text{Liczba wszystkich żądań}}
\]
Im wyższy wskaźnik trafień, tym lepsza efektywność pamięci podręcznej.

\subsection{Pamięć wirtualna}
Pamięć wirtualna to technika pozwalająca na zwiększenie dostępnej pamięci operacyjnej poprzez wykorzystanie przestrzeni na dysku twardym jako dodatkowej pamięci.

\subsubsection{Zasada działania}
Gdy ilość dostępnej pamięci RAM jest niewystarczająca, system operacyjny przenosi rzadko używane strony pamięci na dysk twardy (pliki stronicowania lub wymiany). Proces ten nosi nazwę \textbf{stronicowania}.

\subsubsection{Zalety pamięci wirtualnej}
\begin{itemize}
    \item Pozwala uruchamiać programy wymagające więcej pamięci, niż jest fizycznie dostępne.
    \item Izoluje procesy, zapobiegając ich wzajemnemu nadpisywaniu pamięci.
\end{itemize}

\subsubsection{Wady pamięci wirtualnej}
\begin{itemize}
    \item \textbf{Wolniejsza niż RAM} – dostęp do danych na dysku jest znacznie wolniejszy niż w pamięci operacyjnej.
    \item \textbf{Zjawisko thrashingu} – częste przenoszenie stron między RAM a dyskiem może prowadzić do spadku wydajności.
\end{itemize}

\subsubsection{Mechanizmy zarządzania pamięcią wirtualną}
\begin{itemize}
    \item \textbf{Stronicowanie} – podział pamięci na strony o stałej wielkości, które mogą być przenoszone między RAM a dyskiem.
    \item \textbf{Segmentacja} – dzielenie pamięci na segmenty odpowiadające logicznym częściom programu.
\end{itemize}

\subsection{Podsumowanie}
\begin{itemize}
    \item Hierarchia pamięci obejmuje różne poziomy, od szybkich rejestrów po wolne nośniki zewnętrzne.
    \item Pamięć podręczna redukuje liczbę operacji dostępu do RAM, poprawiając wydajność procesora.
    \item Pamięć wirtualna umożliwia symulację większej pamięci RAM, ale może powodować spadek wydajności z powodu operacji wymiany stron.
    \item Efektywność pamięci zależy od wskaźnika trafień w pamięci podręcznej i optymalnego zarządzania pamięcią wirtualną.
\end{itemize}
