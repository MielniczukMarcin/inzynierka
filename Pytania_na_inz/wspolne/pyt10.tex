\section{Procesy i wątki: definicje, cechy wspólne i różnice, metody tworzenia procesów i wątków w różnych systemach operacyjnych}

\subsection{Definicje}

\subsubsection{Proces}
Proces to program w trakcie wykonywania, który posiada własną przestrzeń adresową oraz zasoby przydzielone przez system operacyjny.

\textbf{Cechy procesu:}
\begin{itemize}
    \item Każdy proces ma własną przestrzeń adresową w pamięci.
    \item Posiada przynajmniej jeden wątek wykonania.
    \item Komunikacja między procesami (\textit{Inter-Process Communication, IPC}) wymaga mechanizmów systemowych (np. kolejki komunikatów, potoki, pamięć współdzielona).
\end{itemize}

\subsubsection{Wątek}
Wątek (ang. \textit{thread}) to najmniejsza jednostka wykonania w obrębie procesu. Wątki współdzielą przestrzeń adresową i zasoby procesu.

\textbf{Cechy wątku:}
\begin{itemize}
    \item Wszystkie wątki w procesie działają w tej samej przestrzeni adresowej.
    \item Wątki mogą współdzielić dane globalne i zasoby.
    \item Komunikacja między wątkami jest szybka, gdyż odbywa się poprzez pamięć współdzieloną.
\end{itemize}

\subsection{Cechy wspólne procesów i wątków}
\begin{itemize}
    \item Zarówno procesy, jak i wątki są jednostkami wykonywania programu.
    \item Mogą być tworzone dynamicznie w trakcie działania systemu.
    \item Współbieżność – systemy wielozadaniowe pozwalają na równoczesne wykonywanie wielu procesów i wątków.
\end{itemize}

\subsection{Różnice między procesami a wątkami}

\begin{table}[h]
    \centering
    \renewcommand{\arraystretch}{1.3} % Poprawia czytelność tabeli
    \begin{tabularx}{\textwidth}{|l|X|X|}
        \hline
        \textbf{Cecha} & \textbf{Proces} & \textbf{Wątek} \\
        \hline
        \textbf{Przestrzeń adresowa} & Oddzielna dla każdego procesu & Współdzielona między wątkami w procesie \\
        \hline
        \textbf{Zasoby} & Każdy proces posiada własne zasoby & Wątki współdzielą zasoby procesu \\
        \hline
        \textbf{Przełączanie kontekstu} & Kosztowne, wymaga przełączenia pamięci & Szybkie, ponieważ przestrzeń adresowa jest wspólna \\
        \hline
        \textbf{Komunikacja} & Wymaga mechanizmów IPC & Łatwa i szybka dzięki pamięci współdzielonej \\
        \hline
        \textbf{Zależność} & Procesy są niezależne & Wątki mogą na siebie oddziaływać \\
        \hline
    \end{tabularx}
    \caption{Porównanie procesów i wątków}
\end{table}


\subsection{Metody tworzenia procesów i wątków w różnych systemach operacyjnych}

\subsubsection{Tworzenie procesów}

\textbf{1. Unix/Linux:}  
Najczęściej stosowaną metodą tworzenia nowego procesu jest funkcja \texttt{fork()}.
\begin{verbatim}
#include <unistd.h>
#include <stdio.h>

int main() {
    pid_t pid = fork();

    if (pid == 0) {
        printf("Proces potomny\n");
    } else {
        printf("Proces rodzicielski\n");
    }
    return 0;
}
\end{verbatim}

Innym podejściem jest \texttt{exec()}, które zastępuje obraz procesu nowym programem.

\textbf{2. Windows:}  
W systemie Windows nowy proces tworzy się za pomocą \texttt{CreateProcess()}.
\begin{verbatim}
#include <windows.h>

int main() {
    STARTUPINFO si = { sizeof(si) };
    PROCESS_INFORMATION pi;

    CreateProcess(NULL, "notepad.exe", NULL, NULL, FALSE, 0, NULL, NULL, &si, &pi);

    WaitForSingleObject(pi.hProcess, INFINITE);
    return 0;
}
\end{verbatim}

\subsubsection{Tworzenie wątków}

\textbf{1. Unix/Linux:}  
Wątki w systemach uniksowych tworzy się za pomocą biblioteki POSIX Threads (pthread):
\begin{verbatim}
#include <pthread.h>
#include <stdio.h>

void* funkcja_watku(void* arg) {
    printf("Wątek uruchomiony\n");
    return NULL;
}

int main() {
    pthread_t watek;
    pthread_create(&watek, NULL, funkcja_watku, NULL);
    pthread_join(watek, NULL);
    return 0;
}
\end{verbatim}

\textbf{2. Windows:}  
W systemie Windows wątki tworzy się za pomocą funkcji \texttt{CreateThread()}:
\begin{verbatim}
#include <windows.h>

DWORD WINAPI funkcja_watku(LPVOID lpParam) {
    printf("Wątek uruchomiony\n");
    return 0;
}

int main() {
    HANDLE watek = CreateThread(NULL, 0, funkcja_watku, NULL, 0, NULL);
    WaitForSingleObject(watek, INFINITE);
    return 0;
}
\end{verbatim}

\subsection{Podsumowanie}
\begin{itemize}
    \item Procesy to niezależne jednostki wykonawcze, posiadające własne zasoby i przestrzeń adresową.
    \item Wątki działają w obrębie procesu i współdzielą jego pamięć oraz zasoby.
    \item Tworzenie nowych procesów w systemach Unix/Linux odbywa się głównie poprzez \texttt{fork()}, a w Windows poprzez \texttt{CreateProcess()}.
    \item Tworzenie wątków w Unix/Linux odbywa się za pomocą biblioteki POSIX Threads (\texttt{pthread}), a w Windows przy użyciu \texttt{CreateThread()}.
    \item Wątki są bardziej efektywne pod względem wydajności niż procesy, ale wymagają synchronizacji, aby uniknąć konfliktów dostępu do zasobów.
\end{itemize}