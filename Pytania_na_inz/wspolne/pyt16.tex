\section{Główne modele procesu wytwarzania oprogramowania}

\subsection{Wprowadzenie}
Model procesu wytwarzania oprogramowania określa metodykę organizacji prac nad projektem informatycznym. Jego celem jest zapewnienie wysokiej jakości produktu, efektywnego zarządzania zasobami oraz minimalizacji ryzyka błędów i opóźnień.

\subsection{Główne modele wytwarzania oprogramowania}

\subsubsection{1. Model kaskadowy (\textit{Waterfall})}
Model kaskadowy to najstarszy i najbardziej klasyczny model wytwarzania oprogramowania, w którym proces przebiega liniowo, etap po etapie.

\textbf{Etapy modelu kaskadowego:}
\begin{enumerate}
    \item Analiza wymagań.
    \item Projektowanie systemu.
    \item Implementacja.
    \item Testowanie.
    \item Wdrożenie.
    \item Utrzymanie.
\end{enumerate}

\textbf{Zalety:}
\begin{itemize}
    \item Jasna struktura i dobrze zdefiniowane etapy.
    \item Łatwość zarządzania projektem.
    \item Dobra dokumentacja techniczna.
\end{itemize}

\textbf{Wady:}
\begin{itemize}
    \item Brak elastyczności – trudność wprowadzania zmian na późniejszych etapach.
    \item Wysokie ryzyko wykrycia błędów dopiero na etapie testowania.
    \item Długi czas realizacji przed dostarczeniem działającego produktu.
\end{itemize}

\subsubsection{2. Model V}
Jest rozszerzeniem modelu kaskadowego, w którym do każdego etapu rozwoju przypisany jest odpowiedni etap testowania.

\textbf{Etapy modelu V:}
\begin{itemize}
    \item \textbf{Faza projektowania:} analiza wymagań, projekt systemu, projektowanie modułów.
    \item \textbf{Faza implementacji:} kodowanie.
    \item \textbf{Faza testowania:} testy jednostkowe, integracyjne, systemowe i akceptacyjne.
\end{itemize}

\textbf{Zalety:}
\begin{itemize}
    \item Testowanie jest zintegrowane z każdym etapem rozwoju.
    \item Wczesne wykrywanie błędów.
\end{itemize}

\textbf{Wady:}
\begin{itemize}
    \item Nadal jest modelem sztywnym, utrudniającym wprowadzanie zmian.
    \item Wymaga dużej ilości dokumentacji.
\end{itemize}

\subsubsection{3. Model iteracyjny}
W modelu iteracyjnym rozwój oprogramowania odbywa się w cyklach (iteracjach), w których dodawane są kolejne funkcjonalności.

\textbf{Zalety:}
\begin{itemize}
    \item Możliwość wczesnego dostarczenia działających fragmentów systemu.
    \item Lepsza adaptacja do zmieniających się wymagań.
\end{itemize}

\textbf{Wady:}
\begin{itemize}
    \item Potrzeba częstej komunikacji z klientem.
    \item Wymaga dobrej organizacji pracy.
\end{itemize}

\subsubsection{4. Model przyrostowy (\textit{Incremental})}
W modelu przyrostowym system jest budowany stopniowo poprzez dodawanie kolejnych modułów.

\textbf{Zalety:}
\begin{itemize}
    \item Klient otrzymuje działające wersje systemu na wczesnych etapach.
    \item Mniejsze ryzyko błędów projektowych.
\end{itemize}

\textbf{Wady:}
\begin{itemize}
    \item Może prowadzić do problemów z integracją modułów.
    \item Wymaga dobrego zarządzania wersjami.
\end{itemize}

\subsubsection{5. Model spiralny}
Łączy elementy modelu kaskadowego i iteracyjnego, koncentrując się na minimalizacji ryzyka poprzez wielokrotne cykle planowania, projektowania, implementacji i testowania.

\textbf{Zalety:}
\begin{itemize}
    \item Dobre zarządzanie ryzykiem.
    \item Elastyczność w dostosowywaniu wymagań.
\end{itemize}

\textbf{Wady:}
\begin{itemize}
    \item Złożoność zarządzania procesem.
    \item Wysoki koszt realizacji.
\end{itemize}

\subsubsection{6. Metodyki zwinne (\textit{Agile})}
Zwinne podejścia do wytwarzania oprogramowania, takie jak \textbf{Scrum} czy \textbf{Kanban}, opierają się na iteracyjnym podejściu, ścisłej współpracy z klientem i szybkim dostarczaniu wartości.

\textbf{Zalety:}
\begin{itemize}
    \item Duża elastyczność w dostosowywaniu projektu do zmieniających się wymagań.
    \item Wczesne dostarczanie wartościowych funkcjonalności.
\end{itemize}

\textbf{Wady:}
\begin{itemize}
    \item Wymaga intensywnej komunikacji i zaangażowania zespołu.
    \item Trudności w planowaniu długoterminowym.
\end{itemize}

\subsection{Podsumowanie}
\begin{itemize}
    \item Model kaskadowy i model V są dobrze ustrukturyzowane, ale mało elastyczne.
    \item Modele iteracyjne i przyrostowe umożliwiają stopniowe wdrażanie funkcjonalności.
    \item Model spiralny zapewnia dobrą kontrolę ryzyka.
    \item Metodyki Agile umożliwiają szybkie reagowanie na zmiany i ścisłą współpracę z klientem.
\end{itemize}
