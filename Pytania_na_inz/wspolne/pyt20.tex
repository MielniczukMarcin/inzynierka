\section{Problem przekleństwa wymiarowości}

\subsection{Wprowadzenie}
Przekleństwo wymiarowości (ang. \textit{curse of dimensionality}) to zjawisko, w którym wzrost liczby wymiarów przestrzeni cech prowadzi do istotnych problemów obliczeniowych, analitycznych oraz modelowania danych. Problemy te występują w uczeniu maszynowym, statystyce, analizie danych i eksploracji dużych zbiorów danych.

\subsection{Przyczyny przekleństwa wymiarowości}
\begin{itemize}
    \item \textbf{Wzrost objętości przestrzeni} – im więcej wymiarów, tym większa przestrzeń do przeszukiwania.
    \item \textbf{Rozrzedzenie danych} – w wysokiej wymiarowości dane stają się rzadkie, co utrudnia efektywne modelowanie.
    \item \textbf{Problemy z odległościami} – wymiary wpływają na metryki odległości, co powoduje, że punkty stają się bardziej równomiernie rozmieszczone.
    \item \textbf{Wzrost złożoności obliczeniowej} – operacje na danych wysokiej wymiarowości wymagają większych zasobów obliczeniowych.
\end{itemize}

\subsection{Konsekwencje przekleństwa wymiarowości}
\subsubsection{1. Problemy w klasyfikacji i klasteryzacji}
Wysoka wymiarowość może prowadzić do sytuacji, w której klasyfikatory i algorytmy klasteryzacji tracą skuteczność, ponieważ różnice między punktami stają się mniej wyraźne.

\subsubsection{2. Wzrost liczby danych wymaganych do efektywnego modelowania}
Dla wysokiej wymiarowości konieczna jest większa liczba próbek, aby uniknąć przeuczenia (ang. \textit{overfitting}).

\subsubsection{3. Problemy w analizie podobieństwa}
W metrykach takich jak odległość euklidesowa większość punktów może znajdować się w podobnych odległościach, co utrudnia identyfikację najbliższych sąsiadów.

\subsection{Metody radzenia sobie z przekleństwem wymiarowości}

\subsubsection{1. Redukcja wymiarowości}
Redukcja liczby wymiarów może poprawić efektywność algorytmów i ich interpretowalność.

\textbf{Przykłady metod:}
\begin{itemize}
    \item \textbf{Główne składowe (PCA)} – transformacja danych w nową przestrzeń o mniejszej liczbie wymiarów.
    \item \textbf{Analiza składowych niezależnych (ICA)} – rozkład danych na komponenty niezależne statystycznie.
    \item \textbf{t-SNE, UMAP} – metody redukcji wymiarowości do wizualizacji danych.
\end{itemize}

\subsubsection{2. Wybór cech (Feature Selection)}
Selekcja najistotniejszych cech pozwala zmniejszyć wymiarowość bez utraty informacji.

\textbf{Metody:}
\begin{itemize}
    \item \textbf{Filtry} – analiza cech na podstawie korelacji, testów statystycznych.
    \item \textbf{Metody osadzone} – np. LASSO, które usuwa mniej istotne cechy podczas uczenia modelu.
\end{itemize}

\subsubsection{3. Użycie algorytmów odpornych na wysoką wymiarowość}
Niektóre algorytmy, np. drzewa decyzyjne czy metody oparte na rzadkich reprezentacjach, mogą lepiej działać w wysokiej wymiarowości.

\subsection{Podsumowanie}
\begin{itemize}
    \item Przekleństwo wymiarowości utrudnia analizę i modelowanie danych, prowadząc do problemów w klasyfikacji, klasteryzacji i eksploracji danych.
    \item Wysoka wymiarowość zwiększa wymagania obliczeniowe i prowadzi do rozrzedzenia danych.
    \item Metody takie jak redukcja wymiarowości (PCA, t-SNE), selekcja cech i dobór odpowiednich algorytmów pomagają radzić sobie z tym problemem.
\end{itemize}
