\section{Metody rozwiązywania układów równań liniowych stosowane do bardzo dużych układów równań (wraz z uzasadnieniem)}

\subsection{Wprowadzenie}
Układy równań liniowych pojawiają się w wielu dziedzinach nauki i techniki, takich jak analiza danych, fizyka, inżynieria czy sztuczna inteligencja. W przypadku bardzo dużych układów, standardowe metody analityczne okazują się niewydajne ze względu na wysoką złożoność obliczeniową i wymagania pamięciowe.

Wyróżnia się dwie główne klasy metod:
\begin{itemize}
    \item \textbf{Metody bezpośrednie} – dają dokładne rozwiązanie w skończonej liczbie kroków, ale mogą być kosztowne obliczeniowo.
    \item \textbf{Metody iteracyjne} – pozwalają uzyskać rozwiązanie przybliżone w sposób efektywny, szczególnie dla układów rzadkich.
\end{itemize}

\subsection{Metody bezpośrednie}

\subsubsection{1. Metoda eliminacji Gaussa}
Metoda ta polega na przekształceniu macierzy współczynników do postaci trójkątnej, a następnie rozwiązaniu układu równań poprzez podstawianie wsteczne.

\textbf{Zalety:}
\begin{itemize}
    \item Dokładne rozwiązanie w skończonej liczbie operacji.
    \item Może być efektywna dla małych i średnich układów.
\end{itemize}

\textbf{Wady:}
\begin{itemize}
    \item Dla dużych układów \( O(n^3) \) operacji sprawia, że metoda staje się niepraktyczna.
    \item Może być niestabilna numerycznie dla źle uwarunkowanych macierzy.
\end{itemize}

\subsubsection{2. Metoda faktoryzacji LU}
Metoda ta polega na rozkładzie macierzy \( A \) na iloczyn macierzy dolnotrójkątnej \( L \) i górnotrójkątnej \( U \), tj. \( A = LU \), co umożliwia szybkie rozwiązanie układu.

\textbf{Zalety:}
\begin{itemize}
    \item Skuteczniejsza od eliminacji Gaussa w przypadku wielokrotnego rozwiązywania układów z tą samą macierzą \( A \).
\end{itemize}

\textbf{Wady:}
\begin{itemize}
    \item Koszt obliczeniowy porównywalny z eliminacją Gaussa (\( O(n^3) \)).
    \item Wymaga pełnej macierzy współczynników, co może być problematyczne dla układów rzadkich.
\end{itemize}

\subsubsection{3. Metoda faktoryzacji Cholesky’ego}
Jest to specjalny przypadek faktoryzacji LU, stosowany dla macierzy symetrycznych i dodatnio określonych. Polega na dekompozycji \( A = LL^T \).

\textbf{Zalety:}
\begin{itemize}
    \item Szybsza niż ogólna faktoryzacja LU (koszt \( O(n^3/3) \)).
\end{itemize}

\textbf{Wady:}
\begin{itemize}
    \item Ograniczone zastosowanie tylko do macierzy symetrycznych dodatnio określonych.
\end{itemize}

\subsection{Metody iteracyjne}
Dla bardzo dużych układów równań, szczególnie gdy macierz jest rzadka, korzystniejsze okazują się metody iteracyjne, które pozwalają uzyskać przybliżone rozwiązanie w krótszym czasie.

\subsubsection{1. Metoda Jacobiego}
Metoda Jacobiego polega na iteracyjnym poprawianiu przybliżeń, bazując na wartości z poprzedniego kroku:
\[
x_i^{(k+1)} = \frac{b_i - \sum_{j \neq i} a_{ij} x_j^{(k)}}{a_{ii}}.
\]

\textbf{Zalety:}
\begin{itemize}
    \item Nadaje się do obliczeń równoległych.
    \item Prosta w implementacji.
\end{itemize}

\textbf{Wady:}
\begin{itemize}
    \item Wolna zbieżność.
    \item Działa tylko dla macierzy o dominującej przekątnej.
\end{itemize}

\subsubsection{2. Metoda Gaussa-Seidla}
Jest modyfikacją metody Jacobiego, w której nowe wartości \( x_i^{(k+1)} \) są wykorzystywane natychmiast w dalszych obliczeniach.

\textbf{Zalety:}
\begin{itemize}
    \item Szybsza zbieżność niż metoda Jacobiego.
\end{itemize}

\textbf{Wady:}
\begin{itemize}
    \item Nie zawsze gwarantuje zbieżność.
\end{itemize}

\subsubsection{3. Metoda gradientu sprzężonego}
Jest jedną z najskuteczniejszych metod iteracyjnych do rzadkich układów równań liniowych. Opiera się na minimalizacji funkcji kwadratowej:
\[
x^{(k+1)} = x^{(k)} + \alpha_k p^{(k)}.
\]

\textbf{Zalety:}
\begin{itemize}
    \item Złożoność \( O(n) \) w wielu przypadkach, co jest dużą poprawą w stosunku do metod bezpośrednich.
    \item Skuteczna dla dużych, rzadkich macierzy.
\end{itemize}

\textbf{Wady:}
\begin{itemize}
    \item Wymaga macierzy symetrycznej i dodatnio określonej.
\end{itemize}

\subsubsection{4. Metody wielosiatkowe}
Metody wielosiatkowe (\textit{Multigrid methods}) rozwiązują problem na różnych poziomach siatki, co przyspiesza zbieżność.

\textbf{Zalety:}
\begin{itemize}
    \item Bardzo szybka zbieżność (często \( O(n) \)).
\end{itemize}

\textbf{Wady:}
\begin{itemize}
    \item Trudność w implementacji i dostosowaniu do konkretnego problemu.
\end{itemize}

\subsection{Podsumowanie}
Wybór metody zależy od charakterystyki układu równań:
\begin{itemize}
    \item Metody bezpośrednie (Gaussa, LU, Cholesky’ego) są dokładne, ale kosztowne obliczeniowo (\( O(n^3) \)).
    \item Metody iteracyjne (Jacobiego, Gaussa-Seidla, gradientu sprzężonego) są efektywne dla dużych, rzadkich układów.
    \item Metody wielosiatkowe są jednymi z najszybszych, ale trudniejsze w implementacji.
\end{itemize}

W praktyce dla bardzo dużych układów równań liniowych najczęściej stosuje się metody iteracyjne, zwłaszcza metodę gradientu sprzężonego i metody wielosiatkowe, ze względu na ich dobrą skalowalność i efektywność pamięciową.
