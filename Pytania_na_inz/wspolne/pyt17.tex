\section{Mechanizmy indeksowania w relacyjnych bazach danych}

\subsection{Wprowadzenie}
Indeksowanie w relacyjnych bazach danych to technika optymalizacyjna, która przyspiesza wyszukiwanie, sortowanie oraz filtrowanie danych. Indeksy są strukturami danych, które umożliwiają szybki dostęp do wierszy tabeli bez konieczności przeszukiwania całej tabeli.

\subsection{Rodzaje indeksów}

\subsubsection{1. Indeks klastrowany (Clustered Index)}
W indeksie klastrowanym dane w tabeli są fizycznie przechowywane w kolejności określonej przez indeks. Każda tabela może mieć tylko jeden indeks klastrowany.

\textbf{Zalety:}
\begin{itemize}
    \item Przyspiesza operacje wyszukiwania i sortowania według klucza indeksu.
    \item Wydajniejszy dla zapytań, które zwracają zakresy danych.
\end{itemize}

\textbf{Wady:}
\begin{itemize}
    \item Wolniejsza operacja wstawiania i aktualizacji, ponieważ może wymagać reorganizacji danych.
    \item Może zajmować więcej miejsca na dysku.
\end{itemize}

\textbf{Przykład w SQL:}
\begin{verbatim}
CREATE CLUSTERED INDEX idx_klastrowany ON Pracownicy (Nazwisko);
\end{verbatim}

\subsubsection{2. Indeks nieklastrowany (Non-clustered Index)}
Indeks nieklastrowany przechowuje wskaźniki do rzeczywistych danych, nie zmieniając ich fizycznego rozmieszczenia.

\textbf{Zalety:}
\begin{itemize}
    \item Można utworzyć wiele indeksów nieklastrowanych dla jednej tabeli.
    \item Przyspiesza wyszukiwanie według wartości, które nie są kluczami głównymi.
\end{itemize}

\textbf{Wady:}
\begin{itemize}
    \item Może spowolnić operacje \texttt{INSERT}, \texttt{UPDATE} i \texttt{DELETE}.
    \item Każdy indeks dodatkowo zużywa przestrzeń dyskową.
\end{itemize}

\textbf{Przykład w SQL:}
\begin{verbatim}
CREATE INDEX idx_nieklastrowany ON Pracownicy (Stanowisko);
\end{verbatim}

\subsubsection{3. Indeks wielokolumnowy (Composite Index)}
Jest to indeks tworzony na więcej niż jednej kolumnie, co przyspiesza wyszukiwanie połączeń między danymi.

\textbf{Zalety:}
\begin{itemize}
    \item Efektywność w zapytaniach, które filtrują dane według kilku kolumn.
\end{itemize}

\textbf{Wady:}
\begin{itemize}
    \item Zapytania muszą używać pierwszej kolumny indeksu, aby indeks był efektywny.
\end{itemize}

\textbf{Przykład w SQL:}
\begin{verbatim}
CREATE INDEX idx_wielokolumnowy ON Pracownicy (Nazwisko, Imie);
\end{verbatim}

\subsubsection{4. Indeks unikalny (Unique Index)}
Indeks, który zapewnia unikalność wartości w danej kolumnie.

\textbf{Zalety:}
\begin{itemize}
    \item Zapobiega duplikacji danych.
\end{itemize}

\textbf{Przykład w SQL:}
\begin{verbatim}
CREATE UNIQUE INDEX idx_unikalny ON Pracownicy (Email);
\end{verbatim}

\subsubsection{5. Indeks pełnotekstowy (Full-text Index)}
Stosowany do wyszukiwania w dużych zbiorach tekstowych.

\textbf{Zastosowanie:}
\begin{itemize}
    \item Wyszukiwanie pełnotekstowe w bazach danych (np. w dokumentach).
\end{itemize}

\textbf{Przykład w SQL Server:}
\begin{verbatim}
CREATE FULLTEXT INDEX ON Dokumenty (Tresc);
\end{verbatim}

\subsection{Podsumowanie}
\begin{itemize}
    \item Indeksy przyspieszają wyszukiwanie danych, ale mogą zwiększyć czas operacji modyfikacji.
    \item Istnieją indeksy klastrowane, nieklastrowane, wielokolumnowe, unikalne i pełnotekstowe.
    \item Wybór odpowiedniego indeksu zależy od charakterystyki danych i częstości wykonywania operacji.
\end{itemize}
