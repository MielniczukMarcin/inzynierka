\section{Rodzaje komunikatów niewerbalnych}

\subsection{Wprowadzenie}
Komunikacja niewerbalna obejmuje wszelkie formy przekazywania informacji, które nie wykorzystują słów. Obejmuje gesty, mimikę, postawę ciała, kontakt wzrokowy, ton głosu oraz inne sposoby wyrażania emocji i intencji.

\subsection{Podstawowe rodzaje komunikacji niewerbalnej}

\subsubsection{1. Mowa ciała (\textit{Kinezjetyka})}
Mowa ciała to ruchy i postawy ciała, które przekazują emocje oraz nastawienie rozmówcy. Do najważniejszych elementów należą:
\begin{itemize}
    \item \textbf{Gesty} – np. kiwanie głową, machanie ręką, skrzyżowanie rąk na klatce piersiowej.
    \item \textbf{Mimika twarzy} – ruchy mięśni twarzy, które odzwierciedlają emocje (np. uśmiech, zmarszczenie brwi).
    \item \textbf{Postawa ciała} – sposób, w jaki stoimy lub siedzimy, może wyrażać otwartość, zamknięcie, pewność siebie lub uległość.
\end{itemize}

\subsubsection{2. Kontakt wzrokowy}
Oczy odgrywają kluczową rolę w komunikacji niewerbalnej. Kontakt wzrokowy może oznaczać:
\begin{itemize}
    \item Zainteresowanie rozmową (utrzymanie kontaktu wzrokowego).
    \item Niepewność lub uległość (unikanie kontaktu wzrokowego).
    \item Dominację lub agresję (intensywne spojrzenie).
\end{itemize}

\subsubsection{3. Proksemika (dystans interpersonalny)}
Proksemika to badanie przestrzeni między rozmówcami. Wyróżnia się cztery strefy dystansu:
\begin{itemize}
    \item \textbf{Strefa intymna} (0–45 cm) – bliskie relacje, rodzina, partnerzy.
    \item \textbf{Strefa osobista} (45–120 cm) – rozmowy z przyjaciółmi i znajomymi.
    \item \textbf{Strefa społeczna} (120–360 cm) – kontakty zawodowe.
    \item \textbf{Strefa publiczna} (powyżej 360 cm) – wykłady, przemówienia.
\end{itemize}

\subsubsection{4. Parajęzyk (cechy wokalne)}
Parajęzyk to sposób, w jaki mówimy, niezależnie od treści wypowiedzi. Obejmuje:
\begin{itemize}
    \item \textbf{Ton głosu} – może wyrażać emocje, np. radość, smutek, złość.
    \item \textbf{Tempo mówienia} – szybkie mówienie może wskazywać na zdenerwowanie, wolne na powagę.
    \item \textbf{Natężenie głosu} – głośność wypowiedzi wpływa na odbiór emocji.
\end{itemize}

\subsubsection{5. Dotyk (\textit{Haptyka})}
Dotyk jest istotnym elementem komunikacji, szczególnie w relacjach międzyludzkich. Może oznaczać:
\begin{itemize}
    \item Sympatię (uścisk dłoni, poklepanie po plecach).
    \item Dominację (silny uścisk dłoni).
    \item Komfort lub pocieszenie (przytulenie).
\end{itemize}

\subsubsection{6. Wygląd zewnętrzny}
Ubiór, fryzura, biżuteria i ogólny wygląd wpływają na sposób postrzegania danej osoby. Może on sugerować status społeczny, profesjonalizm, kreatywność lub przynależność do określonej grupy.

\subsubsection{7. Chronemika (zarządzanie czasem)}
Sposób, w jaki ludzie zarządzają czasem, może być formą komunikacji niewerbalnej:
\begin{itemize}
    \item Punktualność świadczy o szacunku do rozmówcy.
    \item Spóźnienia mogą być odbierane jako brak organizacji lub lekceważenie.
\end{itemize}

\subsubsection{8. Artefakty (przedmioty)}
Przedmioty, które nosimy lub używamy, mogą przekazywać informacje o statusie społecznym, zainteresowaniach lub osobowości. Przykłady:
\begin{itemize}
    \item Markowe ubrania mogą sugerować prestiż.
    \item Kolorystyka i styl biżuterii mogą podkreślać indywidualność.
\end{itemize}

\subsection{Podsumowanie}
Komunikacja niewerbalna odgrywa kluczową rolę w interakcjach międzyludzkich. Jej główne elementy to:
\begin{itemize}
    \item \textbf{Mowa ciała} (gesty, mimika, postawa ciała).
    \item \textbf{Kontakt wzrokowy} (utrzymanie lub unikanie spojrzenia).
    \item \textbf{Proksemika} (dystans interpersonalny).
    \item \textbf{Parajęzyk} (ton głosu, tempo mówienia).
    \item \textbf{Dotyk} (haptyka – uściski dłoni, przytulenia).
    \item \textbf{Wygląd zewnętrzny} (ubiór, fryzura, akcesoria).
    \item \textbf{Chronemika} (sposób zarządzania czasem).
    \item \textbf{Artefakty} (przedmioty symbolizujące status i osobowość).
\end{itemize}

Rozumienie komunikacji niewerbalnej pozwala na lepszą interpretację intencji rozmówców i skuteczniejsze przekazywanie informacji w różnych kontekstach społecznych i zawodowych.
