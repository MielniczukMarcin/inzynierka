\section{Rodzaje symulacji komputerowych; ich charakterystyka i przykłady}

\subsection{Wprowadzenie}
Symulacja komputerowa to proces modelowania rzeczywistych systemów za pomocą programów komputerowych w celu analizy ich zachowania w różnych warunkach. Symulacje znajdują zastosowanie w nauce, inżynierii, ekonomii oraz medycynie.

\subsection{Rodzaje symulacji komputerowych}

\subsubsection{1. Symulacje deterministyczne}
Symulacje deterministyczne to modele, w których dla tych samych warunków początkowych wyniki są zawsze identyczne.

\textbf{Charakterystyka:}
\begin{itemize}
    \item Brak losowych elementów – przebieg symulacji jest w pełni przewidywalny.
    \item Modelowanie procesów fizycznych i technicznych.
\end{itemize}

\textbf{Przykłady:}
\begin{itemize}
    \item Symulacje dynamiki pojazdów (np. analiza ruchu mechanicznego).
    \item Obliczenia przepływu ciepła i mechaniki płynów.
    \item Modele ruchu planet w Układzie Słonecznym.
\end{itemize}

\subsubsection{2. Symulacje stochastyczne}
Symulacje stochastyczne wykorzystują losowość i prawdopodobieństwo do modelowania rzeczywistości.

\textbf{Charakterystyka:}
\begin{itemize}
    \item Wyniki różnią się dla tych samych parametrów wejściowych.
    \item Używane w modelach rzeczywistych, gdzie występuje niepewność.
\end{itemize}

\textbf{Przykłady:}
\begin{itemize}
    \item Modelowanie giełdy papierów wartościowych.
    \item Symulacje epidemiologiczne (np. rozprzestrzenianie chorób).
    \item Metoda Monte Carlo w analizie ryzyka.
\end{itemize}

\subsubsection{3. Symulacje ciągłe}
Modele ciągłe opisują systemy za pomocą równań różniczkowych.

\textbf{Charakterystyka:}
\begin{itemize}
    \item Ciągła zmiana wartości zmiennych w czasie.
    \item Stosowane w naukach przyrodniczych i inżynierii.
\end{itemize}

\textbf{Przykłady:}
\begin{itemize}
    \item Modelowanie klimatu i zmian pogodowych.
    \item Dynamika populacji w ekosystemach.
    \item Symulacje obwodów elektrycznych.
\end{itemize}

\subsubsection{4. Symulacje dyskretne}
Modele dyskretne działają na zdarzeniach zachodzących w określonych momentach czasu.

\textbf{Charakterystyka:}
\begin{itemize}
    \item System zmienia się w określonych punktach czasu.
    \item Wykorzystywane w modelowaniu systemów kolejkowych, produkcyjnych i logistycznych.
\end{itemize}

\textbf{Przykłady:}
\begin{itemize}
    \item Modelowanie przepływu ruchu na skrzyżowaniach.
    \item Symulacja działania systemów produkcyjnych.
    \item Analiza wydajności sieci komputerowych.
\end{itemize}

\subsubsection{5. Symulacje hybrydowe}
Łączą elementy symulacji ciągłej i dyskretnej.

\textbf{Przykłady:}
\begin{itemize}
    \item Modelowanie systemów medycznych (np. układu krwionośnego z impulsami pracy serca).
    \item Symulacje systemów cyber-fizycznych (np. robotyka autonomiczna).
\end{itemize}

\subsubsection{6. Symulacje w czasie rzeczywistym}
Symulacje, w których przetwarzanie danych odbywa się na bieżąco.

\textbf{Przykłady:}
\begin{itemize}
    \item Trenażery lotnicze.
    \item Symulacje jazdy samochodem (np. systemy ADAS).
    \item Gry komputerowe i rzeczywistość wirtualna.
\end{itemize}

\subsection{Podsumowanie}
\begin{itemize}
    \item Symulacje komputerowe pozwalają modelować rzeczywiste procesy i systemy.
    \item Wyróżniamy symulacje deterministyczne, stochastyczne, ciągłe, dyskretne, hybrydowe i w czasie rzeczywistym.
    \item Wybór modelu symulacji zależy od charakterystyki badanego systemu i jego zastosowania.
\end{itemize}
