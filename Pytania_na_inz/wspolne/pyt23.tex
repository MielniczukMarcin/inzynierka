\section{Podstawowe modele kontroli dostępu}

\subsection{Wprowadzenie}
Kontrola dostępu to mechanizm zapewniający bezpieczeństwo systemów informatycznych poprzez ograniczenie dostępu do zasobów. Istnieją różne modele kontroli dostępu, które definiują zasady przyznawania i egzekwowania uprawnień użytkowników.

\subsection{Główne modele kontroli dostępu}

\subsubsection{1. Kontrola dostępu oparta na listach kontroli dostępu (ACL – Access Control List)}
ACL to zbiór reguł określających, kto i w jaki sposób może uzyskać dostęp do danego zasobu.

\textbf{Cechy modelu ACL:}
\begin{itemize}
    \item Każdy zasób ma przypisaną listę użytkowników i ich uprawnień.
    \item Można definiować szczegółowe reguły dostępu dla poszczególnych użytkowników lub grup.
\end{itemize}

\textbf{Przykłady zastosowań:}
\begin{itemize}
    \item Systemy plików (np. NTFS w Windows).
    \item Zapory sieciowe (firewalle).
\end{itemize}

\subsubsection{2. Dyskrecjonalna kontrola dostępu (DAC – Discretionary Access Control)}
DAC pozwala właścicielowi zasobu na dowolne zarządzanie dostępem do niego.

\textbf{Cechy modelu DAC:}
\begin{itemize}
    \item Uprawnienia mogą być przekazywane innym użytkownikom.
    \item Umożliwia elastyczne zarządzanie dostępem.
\end{itemize}

\textbf{Wady modelu DAC:}
\begin{itemize}
    \item Możliwość nieautoryzowanego przekazywania uprawnień.
    \item Brak centralnej kontroli nad dostępem.
\end{itemize}

\textbf{Przykłady zastosowań:}
\begin{itemize}
    \item Systemy operacyjne Windows i Linux.
    \item Systemy bazodanowe.
\end{itemize}

\subsubsection{3. Obowiązkowa kontrola dostępu (MAC – Mandatory Access Control)}
MAC to model, w którym uprawnienia są przydzielane centralnie przez administratora i użytkownicy nie mogą ich zmieniać.

\textbf{Cechy modelu MAC:}
\begin{itemize}
    \item Każdy zasób i użytkownik mają przypisane poziomy klasyfikacji (np. tajne, poufne).
    \item System decyduje, kto może uzyskać dostęp na podstawie polityk bezpieczeństwa.
\end{itemize}

\textbf{Przykłady zastosowań:}
\begin{itemize}
    \item Systemy rządowe i wojskowe.
    \item Bezpieczne systemy operacyjne (np. SELinux).
\end{itemize}

\subsubsection{4. Kontrola dostępu oparta na rolach (RBAC – Role-Based Access Control)}
RBAC przyznaje użytkownikom uprawnienia na podstawie przypisanych im ról.

\textbf{Cechy modelu RBAC:}
\begin{itemize}
    \item Użytkownicy są przypisani do ról, a role posiadają określone uprawnienia.
    \item Ułatwia zarządzanie uprawnieniami w dużych organizacjach.
\end{itemize}

\textbf{Przykłady zastosowań:}
\begin{itemize}
    \item Systemy korporacyjne (np. ERP).
    \item Systemy baz danych i chmury obliczeniowe.
\end{itemize}

\subsubsection{5. Kontrola dostępu oparta na atrybutach (ABAC – Attribute-Based Access Control)}
ABAC przyznaje dostęp na podstawie atrybutów użytkownika, zasobu i kontekstu.

\textbf{Cechy modelu ABAC:}
\begin{itemize}
    \item Uprawnienia są dynamicznie przyznawane na podstawie atrybutów (np. lokalizacja, czas).
    \item Zapewnia wysoką elastyczność i bezpieczeństwo.
\end{itemize}

\textbf{Przykłady zastosowań:}
\begin{itemize}
    \item Zaawansowane systemy bezpieczeństwa w chmurze.
    \item Systemy zgodne z regulacjami (np. HIPAA, GDPR).
\end{itemize}

\subsection{Porównanie modeli kontroli dostępu}

\begin{table}[h]
    \centering
    \renewcommand{\arraystretch}{1.3} % Zwiększa odstępy między wierszami dla lepszej czytelności
    \begin{tabularx}{\textwidth}{|l|X|X|}
        \hline
        \textbf{Model} & \textbf{Cechy} & \textbf{Zastosowanie} \\
        \hline
        ACL  & Lista reguł dla zasobu  & Systemy plików, firewalle \\
        \hline
        DAC  & Właściciel zasobu zarządza dostępem  & Systemy operacyjne, bazy danych \\
        \hline
        MAC  & Centralne zarządzanie dostępem  & Systemy rządowe, wojskowe \\
        \hline
        RBAC & Uprawnienia nadawane według ról  & Organizacje, systemy ERP \\
        \hline
        ABAC & Dynamiczne uprawnienia na podstawie atrybutów  & Systemy chmurowe, zgodność z regulacjami \\
        \hline
    \end{tabularx}
    \caption{Porównanie modeli kontroli dostępu}
\end{table}


\subsection{Podsumowanie}
\begin{itemize}
    \item Kontrola dostępu jest kluczowym elementem bezpieczeństwa systemów informatycznych.
    \item Modele ACL i DAC są stosowane w systemach operacyjnych i bazach danych.
    \item MAC zapewnia najwyższy poziom bezpieczeństwa w systemach rządowych i wojskowych.
    \item RBAC i ABAC są szeroko stosowane w dużych organizacjach i systemach chmurowych.
\end{itemize}
