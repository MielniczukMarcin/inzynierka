\section{Rola i umiejscowienie wywołań systemowych w architekturze systemu operacyjnego, sposób ich uruchamiania, przykłady wywołania}

\subsection{Wprowadzenie}
Wywołania systemowe (ang. \textit{system calls}) to interfejs umożliwiający programom użytkownika komunikację z jądrem systemu operacyjnego. Zapewniają dostęp do zasobów sprzętowych i funkcji systemowych w kontrolowany sposób.

\subsection{Rola wywołań systemowych}
Wywołania systemowe pełnią kluczową rolę w operacjach takich jak:
\begin{itemize}
    \item Zarządzanie procesami (tworzenie, synchronizacja, zakończenie).
    \item Obsługa plików (otwieranie, zamykanie, czytanie, zapis).
    \item Komunikacja międzyprocesowa (IPC).
    \item Zarządzanie pamięcią (alokacja, zwalnianie).
    \item Obsługa urządzeń wejścia/wyjścia.
    \item Sieciowa komunikacja.
\end{itemize}

\subsection{Umiejscowienie wywołań systemowych w architekturze systemu operacyjnego}
W architekturze systemu operacyjnego wywołania systemowe działają na granicy przestrzeni użytkownika (\textit{user space}) i przestrzeni jądra (\textit{kernel space}).

\begin{itemize}
    \item \textbf{Przestrzeń użytkownika} – zawiera procesy aplikacji, które nie mają bezpośredniego dostępu do sprzętu.
    \item \textbf{Przestrzeń jądra} – kontroluje zasoby sprzętowe i wykonuje operacje niskopoziomowe.
    \item \textbf{Interfejs wywołań systemowych} – działa jako pośrednik, przekazując żądania użytkownika do jądra.
\end{itemize}

\subsection{Sposób uruchamiania wywołań systemowych}
Wywołania systemowe są wywoływane przez aplikacje użytkownika w następujący sposób:
\begin{enumerate}
    \item Aplikacja użytkownika wywołuje funkcję biblioteczną (np. \texttt{open()} w języku C).
    \item Funkcja biblioteczna przekazuje żądanie do jądra poprzez instrukcję pułapki (ang. \textit{trap}).
    \item Jądro przełącza kontekst na przestrzeń jądra i wykonuje odpowiednią funkcję.
    \item Wynik jest zwracany do aplikacji użytkownika.
\end{enumerate}

\textbf{Schemat przepływu wywołania systemowego:}
\begin{verbatim}
Program użytkownika → Biblioteka standardowa → Przerwanie → Jądro OS → Wynik
\end{verbatim}

\subsection{Przykłady wywołań systemowych}

\subsubsection{1. Wywołania systemowe do obsługi plików}
\begin{itemize}
    \item \textbf{open()} – otwiera plik.
    \item \textbf{read()} – odczytuje dane z pliku.
    \item \textbf{write()} – zapisuje dane do pliku.
    \item \textbf{close()} – zamyka plik.
\end{itemize}

\textbf{Przykład w języku C:}
\begin{verbatim}
int fd = open("plik.txt", O_RDONLY);
read(fd, buffer, sizeof(buffer));
close(fd);
\end{verbatim}

\subsubsection{2. Wywołania systemowe do zarządzania procesami}
\begin{itemize}
    \item \textbf{fork()} – tworzy nowy proces.
    \item \textbf{exec()} – uruchamia nowy program w bieżącym procesie.
    \item \textbf{wait()} – czeka na zakończenie procesu potomnego.
    \item \textbf{exit()} – kończy działanie procesu.
\end{itemize}

\textbf{Przykład użycia \texttt{fork()}:}
\begin{verbatim}
pid_t pid = fork();
if (pid == 0) {
    printf("Proces potomny\n");
} else {
    printf("Proces macierzysty\n");
}
\end{verbatim}

\subsubsection{3. Wywołania systemowe do zarządzania pamięcią}
\begin{itemize}
    \item \textbf{mmap()} – mapuje plik lub pamięć do przestrzeni adresowej.
    \item \textbf{brk()} – dynamicznie zmienia rozmiar sterty procesu.
\end{itemize}

\textbf{Przykład użycia \texttt{mmap()}:}
\begin{verbatim}
void *ptr = mmap(NULL, 4096, PROT_READ | PROT_WRITE, MAP_PRIVATE | MAP_ANONYMOUS, -1, 0);
\end{verbatim}

\subsubsection{4. Wywołania systemowe do komunikacji międzyprocesowej}
\begin{itemize}
    \item \textbf{pipe()} – tworzy kanał komunikacyjny między procesami.
    \item \textbf{shmget()} – tworzy segment pamięci współdzielonej.
    \item \textbf{socket()} – tworzy gniazdo komunikacyjne.
\end{itemize}

\subsection{Podsumowanie}
\begin{itemize}
    \item Wywołania systemowe są podstawowym mechanizmem komunikacji między aplikacjami a jądrem systemu operacyjnego.
    \item Są używane do zarządzania plikami, procesami, pamięcią oraz komunikacją międzyprocesową.
    \item Wywołania systemowe przechodzą z przestrzeni użytkownika do przestrzeni jądra, zapewniając kontrolowany dostęp do zasobów systemowych.
\end{itemize}
