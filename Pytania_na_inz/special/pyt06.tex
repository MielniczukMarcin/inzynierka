\section{Główne założenia metodyki eXtreme Programming}

\subsection{Wprowadzenie}
eXtreme Programming (XP) to zwinna metodyka wytwarzania oprogramowania, która kładzie nacisk na adaptację do zmieniających się wymagań, wysoką jakość kodu oraz ścisłą współpracę zespołu. XP jest szczególnie skuteczna w dynamicznych projektach, gdzie wymagania klienta mogą ewoluować w trakcie prac.

\subsection{Główne założenia eXtreme Programming}

\subsubsection{1. Komunikacja}
XP promuje intensywną komunikację pomiędzy członkami zespołu, co minimalizuje nieporozumienia i zwiększa efektywność pracy.

\textbf{Przykłady praktyk:}
\begin{itemize}
    \item Codzienne spotkania zespołu (\textit{stand-up meetings}).
    \item Programowanie w parach (\textit{pair programming}).
    \item Bezpośrednie konsultacje z klientem.
\end{itemize}

\subsubsection{2. Prostota}
XP zachęca do stosowania prostych rozwiązań, które spełniają wymagania projektu, eliminując zbędną złożoność.

\textbf{Praktyki:}
\begin{itemize}
    \item Implementowanie tylko niezbędnej funkcjonalności.
    \item Stosowanie przejrzystego i łatwego w utrzymaniu kodu.
\end{itemize}

\subsubsection{3. Informacje zwrotne}
Ciągłe zbieranie informacji zwrotnych pozwala na szybkie dostosowanie systemu do zmieniających się wymagań.

\textbf{Przykłady:}
\begin{itemize}
    \item Testy jednostkowe uruchamiane po każdej zmianie kodu.
    \item Regularne wersje oprogramowania dostarczane klientowi.
\end{itemize}

\subsubsection{4. Odwaga}
Programiści w XP powinni być gotowi na wprowadzanie zmian w kodzie i eliminację błędów bez obawy o negatywne skutki.

\textbf{Praktyki:}
\begin{itemize}
    \item Ciągła refaktoryzacja.
    \item Śmiałe podejmowanie decyzji technicznych.
\end{itemize}

\subsubsection{5. Szacunek}
XP podkreśla znaczenie wzajemnego szacunku w zespole, co wpływa na dobrą atmosferę pracy i skuteczną współpracę.

\textbf{Praktyki:}
\begin{itemize}
    \item Wspólne podejmowanie decyzji projektowych.
    \item Docenianie wkładu każdego członka zespołu.
\end{itemize}

\subsection{Podstawowe praktyki XP}

\subsubsection{1. Programowanie w parach (Pair Programming)}
Dwóch programistów wspólnie pracuje nad tym samym fragmentem kodu, co poprawia jakość i redukuje liczbę błędów.

\subsubsection{2. Ciągła integracja (Continuous Integration)}
Kod jest regularnie integrowany z główną wersją systemu, a testy są automatycznie uruchamiane po każdej zmianie.

\subsubsection{3. Test Driven Development (TDD)}
Najpierw pisane są testy, a dopiero potem implementowana jest funkcjonalność, co zapewnia większą niezawodność systemu.

\subsubsection{4. Refaktoryzacja}
Regularne poprawianie kodu w celu zwiększenia jego czytelności i efektywności.

\subsubsection{5. Małe wydania (Small Releases)}
System jest dostarczany w krótkich iteracjach, co pozwala klientowi na bieżąco oceniać postęp prac.

\subsection{Podsumowanie}
\begin{itemize}
    \item XP koncentruje się na komunikacji, prostocie, informacji zwrotnej, odwadze i szacunku.
    \item Kluczowe praktyki to programowanie w parach, TDD, ciągła integracja i refaktoryzacja.
    \item Metodyka ta pozwala na szybkie reagowanie na zmiany i dostarczanie wysokiej jakości oprogramowania.
\end{itemize}
