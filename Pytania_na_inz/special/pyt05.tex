\section{Refaktoryzacja oprogramowania i wybrane jej sposoby}

\subsection{Wprowadzenie}
Refaktoryzacja oprogramowania to proces restrukturyzacji kodu w celu poprawy jego jakości, czytelności i utrzymania, bez zmiany zewnętrznego zachowania programu. Jest kluczowa dla długoterminowej efektywności projektu i zapobiega zjawisku długu technologicznego.

\subsection{Cele refaktoryzacji}
\begin{itemize}
    \item Poprawa czytelności i zrozumiałości kodu.
    \item Zmniejszenie złożoności i redundancji.
    \item Ułatwienie przyszłego rozwoju i utrzymania systemu.
    \item Poprawa wydajności poprzez eliminację zbędnych operacji.
    \item Zwiększenie testowalności i redukcja błędów.
\end{itemize}

\subsection{Podstawowe techniki refaktoryzacji}

\subsubsection{1. Ekstrakcja metod (\textit{Extract Method})}
Polega na wydzieleniu fragmentu kodu do nowej metody w celu zwiększenia czytelności i unikania duplikacji.

\textbf{Przykład przed refaktoryzacją:}
\begin{verbatim}
void oblicz() {
    int suma = 0;
    for (int i = 0; i < lista.size(); i++) {
        suma += lista[i];
    }
    System.out.println("Suma: " + suma);
}
\end{verbatim}

\textbf{Po refaktoryzacji:}
\begin{verbatim}
void oblicz() {
    int suma = sumujElementy();
    System.out.println("Suma: " + suma);
}

int sumujElementy() {
    int suma = 0;
    for (int i = 0; i < lista.size(); i++) {
        suma += lista[i];
    }
    return suma;
}
\end{verbatim}

\subsubsection{2. Zastąpienie magicznych liczb stałymi (\textit{Replace Magic Number with Constant})}
Pozwala uniknąć nieczytelnych wartości liczbowych w kodzie.

\textbf{Przed:}
\begin{verbatim}
double obliczObwod(double promien) {
    return 2 * 3.14159 * promien;
}
\end{verbatim}

\textbf{Po:}
\begin{verbatim}
static final double PI = 3.14159;

double obliczObwod(double promien) {
    return 2 * PI * promien;
}
\end{verbatim}

\subsubsection{3. Wprowadzenie obiektu parametru (\textit{Introduce Parameter Object})}
Zamiast przekazywać wiele argumentów, można przekazać obiekt enkapsulujący dane.

\textbf{Przed:}
\begin{verbatim}
void ustawWymiary(int szerokosc, int wysokosc, int glebokosc) { ... }
\end{verbatim}

\textbf{Po:}
\begin{verbatim}
class Wymiary {
    int szerokosc, wysokosc, glebokosc;
}

void ustawWymiary(Wymiary wymiary) { ... }
\end{verbatim}

\subsubsection{4. Usunięcie zbędnych komentarzy}
Czysty kod powinien być samodokumentujący się, a nadmiar komentarzy może świadczyć o złej jakości kodu.

\textbf{Przed:}
\begin{verbatim}
// Dodaje produkt do listy
listaProduktow.add(produkt);
\end{verbatim}

\textbf{Po:}
\begin{verbatim}
dodajProduktDoListy(produkt);
\end{verbatim}

\subsubsection{5. Podział dużych klas (\textit{Extract Class})}
Jeśli klasa ma zbyt wiele odpowiedzialności, warto podzielić ją na mniejsze.

\textbf{Przed:}
\begin{verbatim}
class Zamowienie {
    List<Produkt> produkty;
    double obliczCene() { ... }
    void wyslijEmailPotwierdzajacy() { ... }
}
\end{verbatim}

\textbf{Po podziale:}
\begin{verbatim}
class Zamowienie {
    List<Produkt> produkty;
    double obliczCene() { ... }
}

class Notyfikacja {
    void wyslijEmailPotwierdzajacy() { ... }
}
\end{verbatim}

\subsection{Automatyzacja refaktoryzacji}
Wiele narzędzi wspiera refaktoryzację, np.:
\begin{itemize}
    \item IntelliJ IDEA, Eclipse – refaktoryzacja kodu w językach obiektowych.
    \item SonarQube – analiza jakości kodu.
    \item Black, Prettier – formatowanie kodu w Pythonie i JavaScript.
\end{itemize}

\subsection{Podsumowanie}
\begin{itemize}
    \item Refaktoryzacja poprawia jakość kodu i ułatwia jego utrzymanie.
    \item Wprowadzenie metod, eliminacja magicznych liczb i podział dużych klas zwiększają czytelność.
    \item Nowoczesne narzędzia wspomagają proces refaktoryzacji.
    \item Regularna refaktoryzacja zmniejsza ryzyko długu technologicznego.
\end{itemize}
