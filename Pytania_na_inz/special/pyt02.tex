\section{Komunikacja międzyprocesowa z wykorzystaniem pamięci współdzielonej, semaforów i gniazd}

\subsection{Wprowadzenie}
Komunikacja międzyprocesowa (IPC – \textit{Inter-Process Communication}) to mechanizmy umożliwiające wymianę danych i synchronizację między procesami działającymi w tym samym systemie operacyjnym lub w różnych systemach. Istnieje wiele metod IPC, w tym pamięć współdzielona, semafory i gniazda.

\subsection{Pamięć współdzielona (Shared Memory)}
Pamięć współdzielona to mechanizm IPC, który pozwala różnym procesom na bezpośredni dostęp do wspólnego obszaru pamięci.

\textbf{Charakterystyka:}
\begin{itemize}
    \item Umożliwia szybkie przesyłanie danych między procesami.
    \item Procesy muszą synchronizować dostęp do pamięci, aby unikać konfliktów.
\end{itemize}

\textbf{Zalety:}
\begin{itemize}
    \item Bardzo szybka wymiana danych.
    \item Minimalna narzutowa komunikacja.
\end{itemize}

\textbf{Wady:}
\begin{itemize}
    \item Brak mechanizmu synchronizacji – wymaga dodatkowych narzędzi (np. semaforów).
    \item Może prowadzić do problemów z bezpieczeństwem danych.
\end{itemize}

\textbf{Przykład w języku C:}
\begin{verbatim}
int shm_id = shmget(IPC_PRIVATE, 1024, IPC_CREAT | 0666);
char *shm_ptr = (char*) shmat(shm_id, NULL, 0);
\end{verbatim}

\subsection{Semafory}
Semafory to mechanizm synchronizacji procesów, który pozwala na kontrolę dostępu do zasobów współdzielonych.

\textbf{Rodzaje semaforów:}
\begin{itemize}
    \item \textbf{Semafory binarne} – działają jak blokada (0/1).
    \item \textbf{Semafory liczbowe} – pozwalają ograniczyć liczbę jednoczesnych dostępów do zasobu.
\end{itemize}

\textbf{Zastosowanie:}
\begin{itemize}
    \item Synchronizacja dostępu do pamięci współdzielonej.
    \item Unikanie warunków wyścigu.
\end{itemize}

\textbf{Przykład użycia semaforów w C:}
\begin{verbatim}
sem_t sem;
sem_init(&sem, 0, 1);
sem_wait(&sem);
// Sekcja krytyczna
sem_post(&sem);
\end{verbatim}

\subsection{Gniazda (Sockets)}
Gniazda umożliwiają komunikację między procesami działającymi na tym samym lub różnych komputerach poprzez sieć.

\textbf{Rodzaje gniazd:}
\begin{itemize}
    \item \textbf{Gniazda domeny UNIX} – służą do komunikacji międzyprocesowej w jednym systemie operacyjnym.
    \item \textbf{Gniazda sieciowe} – umożliwiają komunikację przez sieć (TCP/UDP).
\end{itemize}

\textbf{Zastosowanie:}
\begin{itemize}
    \item Komunikacja klient-serwer (np. HTTP, FTP).
    \item Wymiana danych między aplikacjami rozproszonymi.
\end{itemize}

\textbf{Przykład w języku C (gniazdo TCP):}
\begin{verbatim}
int sockfd = socket(AF_INET, SOCK_STREAM, 0);
bind(sockfd, (struct sockaddr*)&server_addr, sizeof(server_addr));
listen(sockfd, 5);
int new_sock = accept(sockfd, (struct sockaddr*)&client_addr, &addrlen);
\end{verbatim}

\subsection{Porównanie metod IPC}

\begin{table}[h]
    \centering
    \renewcommand{\arraystretch}{1.3}
    \begin{tabular}{|c|c|c|c|}
        \hline
        \textbf{Metoda} & \textbf{Szybkość} & \textbf{Zastosowanie} & \textbf{Wady} \\
        \hline
        Pamięć współdzielona & Bardzo szybka & Wymiana dużych danych & Brak synchronizacji \\
        \hline
        Semafory & Średnia & Synchronizacja procesów & Możliwe zakleszczenia \\
        \hline
        Gniazda & Wolniejsza & Komunikacja sieciowa & Większy narzut \\
        \hline
    \end{tabular}
    \caption{Porównanie metod IPC}
\end{table}

\subsection{Podsumowanie}
\begin{itemize}
    \item Pamięć współdzielona jest szybka, ale wymaga synchronizacji.
    \item Semafory kontrolują dostęp do zasobów i zapobiegają konfliktom.
    \item Gniazda pozwalają na komunikację między procesami w różnych systemach.
    \item Wybór metody IPC zależy od potrzeb aplikacji i środowiska jej działania.
\end{itemize}
