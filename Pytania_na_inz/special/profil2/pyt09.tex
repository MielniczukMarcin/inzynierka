\section{Pojęcie skalowalności w przetwarzaniu rozproszonym}

\subsection{Wprowadzenie}
Skalowalność to zdolność systemu do utrzymania lub zwiększenia wydajności wraz ze wzrostem liczby zasobów lub obciążenia. W kontekście przetwarzania rozproszonego oznacza to możliwość dodawania nowych węzłów do systemu bez znaczącego pogorszenia jego efektywności.

\subsection{Rodzaje skalowalności}

\subsubsection{1. Skalowalność wertykalna (pionowa, \textit{vertical scaling})}
Polega na zwiększaniu mocy pojedynczego węzła poprzez dodanie lepszego sprzętu (więcej pamięci RAM, szybszy procesor, wydajniejszy dysk).

\textbf{Cechy:}
\begin{itemize}
    \item Prostota – zmiana konfiguracji pojedynczego serwera.
    \item Ograniczona skalowalność – istnieje granica sprzętowa dla jednego węzła.
    \item Wysokie koszty – im mocniejszy sprzęt, tym większy koszt jednostkowy.
\end{itemize}

\textbf{Zastosowanie:}
\begin{itemize}
    \item Bazy danych o wysokich wymaganiach sprzętowych.
    \item Systemy wymagające niskich opóźnień i wysokiej wydajności na jednym węźle.
\end{itemize}

\subsubsection{2. Skalowalność horyzontalna (pozioma, \textit{horizontal scaling})}
Polega na dodawaniu nowych węzłów do systemu w celu rozłożenia obciążenia.

\textbf{Cechy:}
\begin{itemize}
    \item Dobra skalowalność – system może obsługiwać większą liczbę użytkowników przez dodanie kolejnych serwerów.
    \item Wymaga zarządzania rozproszonymi zasobami.
    \item Zwykle tańsza w długoterminowej perspektywie niż skalowanie pionowe.
\end{itemize}

\textbf{Zastosowanie:}
\begin{itemize}
    \item Chmura obliczeniowa (np. AWS, Google Cloud).
    \item Systemy Big Data (np. Hadoop, Apache Spark).
    \item Aplikacje webowe o wysokim ruchu (np. serwery CDN, load balancing).
\end{itemize}

\subsubsection{3. Skalowalność funkcjonalna}
Odnosi się do możliwości dodawania nowych funkcji do systemu bez wpływu na jego stabilność i wydajność.

\textbf{Przykłady:}
\begin{itemize}
    \item Mikroserwisy – dodawanie nowych komponentów bez przerywania działania innych usług.
    \item Architektura modułowa w aplikacjach webowych.
\end{itemize}

\subsubsection{4. Skalowalność geograficzna}
Dotyczy systemów rozproszonych działających w różnych lokalizacjach.

\textbf{Zastosowanie:}
\begin{itemize}
    \item Sieci CDN do dostarczania treści w różnych regionach świata.
    \item Rozproszone bazy danych (np. Google Spanner).
\end{itemize}

\subsection{Metody poprawy skalowalności}

\subsubsection{1. Load Balancing (Równoważenie obciążenia)}
Polega na dynamicznym rozkładaniu ruchu pomiędzy wiele serwerów.

\textbf{Techniki:}
\begin{itemize}
    \item Round Robin – kolejność przydzielania zapytań do serwerów.
    \item Least Connections – kierowanie zapytań do najmniej obciążonego serwera.
    \item IP Hash – przypisanie użytkownika do konkretnego serwera na podstawie adresu IP.
\end{itemize}

\subsubsection{2. Sharding (Podział danych)}
Technika polegająca na podziale bazy danych na mniejsze fragmenty (\textit{shardy}), które są przechowywane na różnych serwerach.

\textbf{Przykład:}
\begin{itemize}
    \item Podział użytkowników serwisu społecznościowego według regionów.
\end{itemize}

\subsubsection{3. Caching (Buforowanie)}
Przechowywanie często używanych danych w szybkiej pamięci w celu zmniejszenia obciążenia głównych zasobów.

\textbf{Przykłady:}
\begin{itemize}
    \item Redis – buforowanie zapytań do bazy danych.
    \item Content Delivery Networks (CDN) – przechowywanie treści stron internetowych w wielu lokalizacjach.
\end{itemize}

\subsubsection{4. Asynchroniczna komunikacja}
Umożliwia efektywniejszą wymianę danych między komponentami systemu.

\textbf{Przykłady:}
\begin{itemize}
    \item Kolejki wiadomości (RabbitMQ, Apache Kafka).
    \item Mechanizmy Publish-Subscribe (Pub/Sub).
\end{itemize}

\subsection{Porównanie skalowalności pionowej i poziomej}

\begin{table}[h]
    \centering
    \renewcommand{\arraystretch}{1.3} % Poprawia czytelność tabeli
    \begin{tabularx}{\textwidth}{|l|X|X|}
        \hline
        \textbf{Cecha} & \textbf{Skalowalność pionowa} & \textbf{Skalowalność pozioma} \\
        \hline
        \textbf{Metoda} & Zwiększanie mocy jednego węzła & Dodawanie nowych węzłów \\
        \hline
        \textbf{Koszt} & Wysoki przy dużej rozbudowie & Można skalować stopniowo \\
        \hline
        \textbf{Skalowalność} & Ograniczona przez sprzęt & Praktycznie nieograniczona \\
        \hline
        \textbf{Awaryjność} & Awaria jednego węzła zatrzymuje system & Awaria pojedynczego węzła nie wpływa na całość \\
        \hline
        \textbf{Zastosowanie} & Systemy bazodanowe, aplikacje monolityczne & Chmura, mikroserwisy, aplikacje webowe \\
        \hline
    \end{tabularx}
    \caption{Porównanie skalowalności pionowej i poziomej}
\end{table}


\subsection{Podsumowanie}
\begin{itemize}
    \item Skalowalność to zdolność systemu do utrzymania wydajności przy wzroście obciążenia.
    \item Skalowalność pionowa polega na ulepszaniu pojedynczych węzłów, podczas gdy pozioma na dodawaniu kolejnych jednostek.
    \item Mechanizmy takie jak load balancing, sharding, caching i asynchroniczna komunikacja pozwalają na efektywne zarządzanie rosnącym obciążeniem.
    \item Wybór metody skalowania zależy od specyfiki systemu – aplikacje chmurowe preferują skalowanie poziome, a aplikacje wymagające dużej mocy obliczeniowej mogą korzystać ze skalowania pionowego.
\end{itemize}
