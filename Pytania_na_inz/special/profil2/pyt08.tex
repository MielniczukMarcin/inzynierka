\section{Systemy ściśle i luźno powiązane}

\subsection{Wprowadzenie}
Systemy komputerowe mogą być klasyfikowane na podstawie stopnia powiązania ich jednostek obliczeniowych. Wyróżnia się dwa główne modele: \textbf{systemy ściśle powiązane} (\textit{tightly coupled systems}) oraz \textbf{systemy luźno powiązane} (\textit{loosely coupled systems}). Wybór odpowiedniego modelu zależy od wymagań dotyczących wydajności, komunikacji oraz skalowalności.

\subsection{Systemy ściśle powiązane (Tightly Coupled Systems)}

\subsubsection{Charakterystyka}
\begin{itemize}
    \item Współdzielona pamięć – wszystkie jednostki obliczeniowe korzystają ze wspólnej przestrzeni adresowej.
    \item Niski czas dostępu do pamięci – szybka komunikacja między procesorami.
    \item Zarządzanie procesami realizowane przez jeden system operacyjny.
    \item Mocno zintegrowane jednostki przetwarzające.
\end{itemize}

\subsubsection{Przykłady}
\begin{itemize}
    \item Systemy wieloprocesorowe (SMP – Symmetric Multiprocessing).
    \item Superkomputery z pamięcią współdzieloną.
    \item Klasyczne serwery wieloprocesorowe.
\end{itemize}

\subsubsection{Zalety}
\begin{itemize}
    \item Szybka komunikacja między procesorami.
    \item Wysoka wydajność w aplikacjach wymagających częstej wymiany danych.
    \item Jednolity dostęp do zasobów.
\end{itemize}

\subsubsection{Wady}
\begin{itemize}
    \item Ograniczona skalowalność – dodanie kolejnych procesorów prowadzi do problemów z dostępem do pamięci.
    \item Możliwe wąskie gardła komunikacyjne.
    \item Wysoki koszt implementacji.
\end{itemize}

\subsection{Systemy luźno powiązane (Loosely Coupled Systems)}

\subsubsection{Charakterystyka}
\begin{itemize}
    \item Każdy węzeł posiada własną pamięć lokalną i zasoby obliczeniowe.
    \item Komunikacja odbywa się przez sieć komputerową (np. Ethernet, InfiniBand).
    \item Może obejmować systemy działające pod kontrolą różnych systemów operacyjnych.
\end{itemize}

\subsubsection{Przykłady}
\begin{itemize}
    \item Klastry obliczeniowe (np. Beowulf Cluster).
    \item Systemy gridowe (np. Grid Computing).
    \item Chmurowe systemy obliczeniowe (np. Amazon AWS, Google Cloud).
\end{itemize}

\subsubsection{Zalety}
\begin{itemize}
    \item Wysoka skalowalność – można łatwo dodawać kolejne węzły.
    \item Lepsza odporność na awarie – pojedynczy węzeł może się wyłączyć bez zatrzymania całego systemu.
    \item Możliwość wykorzystania heterogenicznych zasobów (różne procesory, systemy operacyjne).
\end{itemize}

\subsubsection{Wady}
\begin{itemize}
    \item Wyższe opóźnienia komunikacyjne w porównaniu do systemów ściśle powiązanych.
    \item Konieczność zarządzania rozproszonymi zasobami.
    \item Problemy z równomiernym podziałem obciążeń.
\end{itemize}

\subsection{Porównanie systemów ściśle i luźno powiązanych}

\begin{table}[h]
    \centering
    \renewcommand{\arraystretch}{1.3}
    \begin{tabular}{|c|c|c|}
        \hline
        \textbf{Cecha} & \textbf{Systemy ściśle powiązane} & \textbf{Systemy luźno powiązane} \\
        \hline
        Pamięć & Współdzielona & Każdy węzeł ma własną \\
        \hline
        Komunikacja & Szybka, poprzez magistralę & Wolniejsza, przez sieć \\
        \hline
        Skalowalność & Ograniczona & Wysoka \\
        \hline
        Odporność na awarie & Niska – awaria procesora wpływa na system & Wysoka – awaria pojedynczego węzła nie zatrzymuje systemu \\
        \hline
        Koszt & Wysoki & Niższy \\
        \hline
        Przykłady & Superkomputery, serwery SMP & Klastry, chmura, systemy gridowe \\
        \hline
    \end{tabular}
    \caption{Porównanie systemów ściśle i luźno powiązanych}
\end{table}

\subsection{Podsumowanie}
\begin{itemize}
    \item Systemy ściśle powiązane oferują wysoką wydajność i niskie opóźnienia, ale mają ograniczoną skalowalność.
    \item Systemy luźno powiązane są bardziej elastyczne i skalowalne, ale wymagają efektywnych mechanizmów komunikacji i synchronizacji.
    \item Wybór odpowiedniego modelu zależy od specyfiki aplikacji – systemy ściśle powiązane są preferowane w obliczeniach naukowych, a systemy luźno powiązane w aplikacjach chmurowych i rozproszonych.
\end{itemize}
