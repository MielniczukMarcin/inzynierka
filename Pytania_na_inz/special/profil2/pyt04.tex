\section{Definicja programowania równoległego i współbieżnego; popularne języki; biblioteki i API}

\subsection{Wprowadzenie}
Programowanie równoległe i współbieżne to dwa podejścia do realizacji obliczeń wielowątkowych, które różnią się sposobem wykonywania zadań i organizacją pracy procesów.

\subsection{Definicja programowania równoległego i współbieżnego}

\subsubsection{1. Programowanie równoległe (Parallel Computing)}
Programowanie równoległe polega na jednoczesnym wykonywaniu wielu operacji w celu przyspieszenia obliczeń.

\textbf{Cechy:}
\begin{itemize}
    \item Wiele zadań wykonywanych jednocześnie na różnych procesorach lub rdzeniach.
    \item Typowe w obliczeniach naukowych, grafice komputerowej i AI.
    \item Wymaga synchronizacji i zarządzania współdzielonymi zasobami.
\end{itemize}

\textbf{Przykłady zastosowań:}
\begin{itemize}
    \item Przetwarzanie dużych zbiorów danych (Big Data).
    \item Algorytmy sztucznej inteligencji (trenowanie sieci neuronowych).
    \item Obliczenia numeryczne i symulacje fizyczne.
\end{itemize}

\subsubsection{2. Programowanie współbieżne (Concurrent Computing)}
Programowanie współbieżne umożliwia wykonywanie wielu zadań w jednym czasie, ale niekoniecznie równolegle.

\textbf{Cechy:}
\begin{itemize}
    \item Wiele procesów może być wykonywanych przeplatająco na jednym rdzeniu.
    \item Stosowane w systemach operacyjnych i aplikacjach sieciowych.
    \item Może wykorzystywać mechanizmy wątków, procesów i asynchronicznego wykonywania kodu.
\end{itemize}

\textbf{Przykłady zastosowań:}
\begin{itemize}
    \item Obsługa wielu klientów w serwerach HTTP.
    \item Zarządzanie zadaniami w systemie operacyjnym.
    \item Aplikacje działające w tle (np. pobieranie plików).
\end{itemize}

\subsection{Popularne języki programowania do programowania równoległego i współbieżnego}

\begin{itemize}
    \item \textbf{C/C++} – języki niskopoziomowe, szeroko stosowane w aplikacjach wysokowydajnościowych.
    \item \textbf{Python} – wspiera współbieżność dzięki bibliotekom takim jak \texttt{threading} i \texttt{multiprocessing}.
    \item \textbf{Java} – posiada wbudowaną obsługę wielowątkowości (\texttt{java.util.concurrent}).
    \item \textbf{Go} – zoptymalizowany pod kątem współbieżności dzięki lekkim wątkom (\textit{goroutines}).
    \item \textbf{Rust} – zapewnia bezpieczne programowanie równoległe, eliminując błędy związane z dostępem do pamięci.
    \item \textbf{CUDA C/C++} – umożliwia programowanie równoległe na GPU.
\end{itemize}

\subsection{Biblioteki i API do programowania równoległego i współbieżnego}

\subsubsection{1. Wątki i synchronizacja}
\begin{itemize}
    \item \textbf{POSIX Threads (pthread)} – standardowa biblioteka w C do obsługi wątków.
    \item \textbf{std::thread} (C++) – biblioteka do obsługi wątków w nowoczesnym C++.
    \item \textbf{Java Threads} – obsługa wielowątkowości wbudowana w język Java.
    \item \textbf{asyncio (Python)} – obsługa asynchronicznych operacji wejścia/wyjścia.
\end{itemize}

\subsubsection{2. Równoległe przetwarzanie na wielu rdzeniach CPU}
\begin{itemize}
    \item \textbf{OpenMP} – standard do równoległego programowania w C, C++ i Fortranie.
    \item \textbf{TBB (Threading Building Blocks)} – biblioteka firmy Intel do równoległego programowania w C++.
    \item \textbf{multiprocessing (Python)} – umożliwia równoległe wykonywanie kodu na wielu procesorach.
\end{itemize}

\subsubsection{3. Programowanie na GPU}
\begin{itemize}
    \item \textbf{CUDA} – platforma NVIDIA do programowania na GPU.
    \item \textbf{OpenCL} – otwarty standard do obliczeń równoległych na różnych platformach.
    \item \textbf{HIP (Heterogeneous-Compute Interface for Portability)} – API firmy AMD do programowania GPU.
\end{itemize}

\subsubsection{4. Programowanie rozproszone}
\begin{itemize}
    \item \textbf{MPI (Message Passing Interface)} – standard komunikacji między procesami w klastrach komputerowych.
    \item \textbf{Apache Spark} – framework do obliczeń rozproszonych na dużych zbiorach danych.
    \item \textbf{Dask (Python)} – narzędzie do obliczeń równoległych na dużych zbiorach danych.
\end{itemize}

\subsection{Podsumowanie}
\begin{itemize}
    \item Programowanie równoległe pozwala na jednoczesne wykonywanie wielu operacji na różnych procesorach lub rdzeniach.
    \item Programowanie współbieżne umożliwia wykonywanie wielu zadań jednocześnie, ale nie zawsze równolegle.
    \item Wiele języków programowania (C++, Python, Java, Go, CUDA) wspiera techniki wielowątkowe i równoległe.
    \item Biblioteki i API, takie jak OpenMP, MPI, CUDA, OpenCL i \texttt{pthread}, ułatwiają implementację obliczeń równoległych i współbieżnych.
\end{itemize}
