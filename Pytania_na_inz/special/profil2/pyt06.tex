\section{Mechanizmy przesyłania komunikatów w systemach rozproszonych}

\subsection{Wprowadzenie}
Systemy rozproszone składają się z wielu węzłów komunikujących się poprzez sieć. Podstawowym mechanizmem wymiany informacji między procesami jest przesyłanie komunikatów (\textit{message passing}). Efektywna komunikacja w systemach rozproszonych wymaga mechanizmów zapewniających niezawodność, synchronizację i efektywne zarządzanie przesyłanymi danymi.

\subsection{Podstawowe modele komunikacji}

\subsubsection{1. Komunikacja synchroniczna i asynchroniczna}
\begin{itemize}
    \item \textbf{Komunikacja synchroniczna} – nadawca czeka na potwierdzenie odbioru komunikatu przed kontynuacją pracy.
    \item \textbf{Komunikacja asynchroniczna} – nadawca wysyła komunikat i natychmiast kontynuuje działanie, a odbiorca może go odebrać później.
\end{itemize}

\textbf{Zastosowanie:}
\begin{itemize}
    \item Synchroniczna – systemy wymagające spójności, np. systemy bankowe.
    \item Asynchroniczna – systemy wymagające wysokiej przepustowości, np. przesyłanie wiadomości w mediach społecznościowych.
\end{itemize}

\subsubsection{2. Komunikacja jednokierunkowa i dwukierunkowa}
\begin{itemize}
    \item \textbf{Jednokierunkowa (Unidirectional)} – komunikacja odbywa się w jednym kierunku (np. \textit{UDP}).
    \item \textbf{Dwukierunkowa (Bidirectional)} – komunikacja wymaga odpowiedzi od odbiorcy (np. \textit{TCP}).
\end{itemize}

\subsubsection{3. Komunikacja jedno-do-jednego i jedno-do-wielu}
\begin{itemize}
    \item \textbf{Jedno-do-jednego (point-to-point)} – komunikacja między dwoma procesami.
    \item \textbf{Jedno-do-wielu (multicast, broadcast)} – komunikacja do wielu procesów jednocześnie.
\end{itemize}

\subsection{Mechanizmy przesyłania komunikatów}

\subsubsection{1. Gniazda (Sockets)}
Gniazda to podstawowy mechanizm komunikacji między procesami w sieci.

\textbf{Rodzaje gniazd:}
\begin{itemize}
    \item \textbf{Gniazda strumieniowe (TCP)} – zapewniają niezawodny, uporządkowany przesył danych.
    \item \textbf{Gniazda datagramowe (UDP)} – zapewniają szybki, ale niegwarantowany przesył danych.
\end{itemize}

\textbf{Przykład kodu w C (gniazdo TCP):}
\begin{verbatim}
int sockfd = socket(AF_INET, SOCK_STREAM, 0);
connect(sockfd, (struct sockaddr*)&server_addr, sizeof(server_addr));
send(sockfd, message, strlen(message), 0);
\end{verbatim}

\subsubsection{2. Kolejki komunikatów (Message Queues)}
Mechanizm kolejek pozwala na przechowywanie i odbieranie komunikatów asynchronicznie.

\textbf{Popularne implementacje:}
\begin{itemize}
    \item \textbf{POSIX Message Queues} – standard w systemach UNIX.
    \item \textbf{RabbitMQ, Apache Kafka} – systemy kolejkowania komunikatów w systemach rozproszonych.
\end{itemize}

\subsubsection{3. Pamięć współdzielona}
Procesy mogą wymieniać dane poprzez wspólną przestrzeń adresową.

\textbf{Zalety:}
\begin{itemize}
    \item Bardzo szybka wymiana danych.
    \item Unikanie narzutu komunikacyjnego sieci.
\end{itemize}

\textbf{Wady:}
\begin{itemize}
    \item Wymaga synchronizacji dostępu (semafory, muteksy).
    \item Ograniczona do systemów działających na jednym węźle.
\end{itemize}

\subsubsection{4. Remote Procedure Call (RPC)}
RPC pozwala na wywoływanie funkcji na zdalnym komputerze tak, jakby były lokalne.

\textbf{Przykłady technologii:}
\begin{itemize}
    \item \textbf{gRPC} – nowoczesne RPC oparte na HTTP/2.
    \item \textbf{XML-RPC, JSON-RPC} – lekkie protokoły RPC.
\end{itemize}

\textbf{Przykład RPC w Pythonie (gRPC):}
\begin{verbatim}
import grpc
channel = grpc.insecure_channel('localhost:50051')
response = stub.MethodName(request)
\end{verbatim}

\subsubsection{5. Publish-Subscribe (Pub/Sub)}
Wzorzec komunikacyjny, w którym nadawca (\textit{publisher}) wysyła komunikaty do kanału, a subskrybenci (\textit{subscribers}) je odbierają.

\textbf{Przykłady implementacji:}
\begin{itemize}
    \item \textbf{Apache Kafka} – skalowalna platforma do przesyłania strumieni danych.
    \item \textbf{MQTT} – protokół komunikacji w IoT.
\end{itemize}

\subsubsection{6. Strumieniowanie komunikatów (Message Streaming)}
Mechanizm umożliwia przesyłanie dużych ilości danych w czasie rzeczywistym.

\textbf{Przykłady implementacji:}
\begin{itemize}
    \item \textbf{Apache Kafka, Apache Pulsar} – rozproszone przetwarzanie strumieniowe.
    \item \textbf{Google Cloud Pub/Sub} – system do strumieniowego przesyłania danych w chmurze.
\end{itemize}

\subsection{Porównanie mechanizmów przesyłania komunikatów}

\begin{table}[h]
    \centering
    \renewcommand{\arraystretch}{1.3} % Poprawia czytelność tabeli
    \begin{tabularx}{\textwidth}{|l|X|X|}
        \hline
        \textbf{Mechanizm} & \textbf{Zalety} & \textbf{Wady} \\
        \hline
        \textbf{Gniazda (Sockets)} & Niski narzut, szybka komunikacja & Wymaga zarządzania połączeniami \\
        \hline
        \textbf{Kolejki komunikatów} & Buforowanie komunikatów & Większe opóźnienia \\
        \hline
        \textbf{Pamięć współdzielona} & Najszybsza wymiana danych & Ograniczona do jednego systemu \\
        \hline
        \textbf{RPC} & Transparentność wywołań & Opóźnienia sieciowe \\
        \hline
        \textbf{Publish-Subscribe} & Łatwa skalowalność & Opóźnienia propagacji \\
        \hline
        \textbf{Strumieniowanie} & Przetwarzanie w czasie rzeczywistym & Złożoność implementacji \\
        \hline
    \end{tabularx}
    \caption{Porównanie mechanizmów przesyłania komunikatów}
\end{table}


\subsection{Podsumowanie}
\begin{itemize}
    \item Przesyłanie komunikatów w systemach rozproszonych może odbywać się synchronicznie lub asynchronicznie.
    \item Gniazda są podstawowym mechanizmem komunikacji w sieci.
    \item Kolejki komunikatów i strumieniowanie danych zapewniają niezależność nadawcy i odbiorcy.
    \item RPC upraszcza wywołania zdalnych funkcji, a Pub/Sub umożliwia komunikację wielu węzłów.
    \item Wybór mechanizmu zależy od wymagań dotyczących niezawodności, opóźnień i skalowalności systemu.
\end{itemize}
