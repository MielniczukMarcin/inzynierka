\section{Topologie systemów rozproszonych}

\subsection{Wprowadzenie}
Systemy rozproszone składają się z wielu węzłów komunikujących się w celu realizacji wspólnych zadań. Struktura połączeń między tymi węzłami, określana jako \textbf{topologia}, wpływa na wydajność, odporność na błędy oraz możliwości skalowania systemu. Wybór topologii zależy od charakterystyki aplikacji i wymagań dotyczących przepustowości, niezawodności i latencji komunikacyjnej.

\subsection{Podstawowe topologie systemów rozproszonych}

\subsubsection{1. Topologia magistrali (Bus)}
Wszystkie węzły są podłączone do jednej wspólnej linii komunikacyjnej (\textit{magistrali}).

\textbf{Cechy:}
\begin{itemize}
    \item Prosta implementacja i niski koszt.
    \item Ograniczona skalowalność – większa liczba węzłów prowadzi do przeciążeń komunikacyjnych.
    \item Awaria magistrali powoduje zatrzymanie całego systemu.
\end{itemize}

\textbf{Zastosowanie:}
\begin{itemize}
    \item Systemy lokalne (LAN) o niewielkiej liczbie węzłów.
    \item Wczesne architektury systemów rozproszonych.
\end{itemize}

\subsubsection{2. Topologia pierścienia (Ring)}
Węzły są połączone w zamknięty łańcuch, gdzie każdy węzeł jest połączony z dwoma sąsiadami.

\textbf{Cechy:}
\begin{itemize}
    \item Efektywne wykorzystanie przepustowości.
    \item Brak kolizji – komunikacja odbywa się w jednym kierunku.
    \item Awaria jednego węzła może przerwać komunikację, jeśli nie zastosowano mechanizmów redundancji.
\end{itemize}

\textbf{Zastosowanie:}
\begin{itemize}
    \item Sieci Token Ring.
    \item Rozproszone systemy pamięci masowej.
\end{itemize}

\subsubsection{3. Topologia gwiazdy (Star)}
Wszystkie węzły są połączone z jednym centralnym węzłem.

\textbf{Cechy:}
\begin{itemize}
    \item Centralizacja ułatwia zarządzanie ruchem sieciowym.
    \item Awaria centralnego węzła powoduje utratę komunikacji całej sieci.
\end{itemize}

\textbf{Zastosowanie:}
\begin{itemize}
    \item Sieci Ethernet z przełącznikami.
    \item Systemy serwer-klient.
\end{itemize}

\subsubsection{4. Topologia drzewa (Tree)}
Hierarchiczna struktura, w której węzły są ułożone w postaci drzewa z centralnym węzłem głównym.

\textbf{Cechy:}
\begin{itemize}
    \item Lepsza skalowalność niż topologia gwiazdy.
    \item Awaria węzła nadrzędnego może odciąć część systemu.
\end{itemize}

\textbf{Zastosowanie:}
\begin{itemize}
    \item Rozproszone systemy plików (np. Hadoop HDFS).
    \item Sieci komunikacyjne.
\end{itemize}

\subsubsection{5. Topologia siatki (Mesh)}
Każdy węzeł jest połączony z kilkoma innymi, tworząc gęsto połączoną sieć.

\textbf{Cechy:}
\begin{itemize}
    \item Wysoka odporność na awarie – redundancja połączeń.
    \item Duże koszty implementacji i zarządzania.
\end{itemize}

\textbf{Zastosowanie:}
\begin{itemize}
    \item Sieci bezprzewodowe (np. Mesh WiFi).
    \item Systemy rozproszone o wysokiej dostępności.
\end{itemize}

\subsubsection{6. Topologia hybrydowa (Hybrid)}
Łączy cechy różnych topologii w celu uzyskania lepszej skalowalności i niezawodności.

\textbf{Zastosowanie:}
\begin{itemize}
    \item Internet i sieci rozproszone na dużą skalę.
    \item Architektury chmurowe.
\end{itemize}

\subsection{Porównanie topologii}

\begin{table}[h]
    \centering
    \renewcommand{\arraystretch}{1.3}
    \begin{tabular}{|c|c|c|c|}
        \hline
        \textbf{Topologia} & \textbf{Zalety} & \textbf{Wady} & \textbf{Zastosowanie} \\
        \hline
        Magistrala & Prosta, tani koszt & Słaba skalowalność & Małe sieci LAN \\
        \hline
        Pierścień & Brak kolizji & Awaria może zatrzymać sieć & Sieci Token Ring \\
        \hline
        Gwiazda & Łatwa administracja & Awaria serwera wyłącza system & Sieci Ethernet \\
        \hline
        Drzewo & Skalowalne & Awaria węzła nadrzędnego izoluje część systemu & HDFS, sieci chmurowe \\
        \hline
        Siatka & Wysoka odporność & Wysoki koszt implementacji & Systemy wysokiej dostępności \\
        \hline
        Hybrydowa & Elastyczność i skalowalność & Złożoność konfiguracji & Internet, chmura \\
        \hline
    \end{tabular}
    \caption{Porównanie topologii systemów rozproszonych}
\end{table}

\subsection{Podsumowanie}
\begin{itemize}
    \item Wybór topologii systemu rozproszonego wpływa na jego wydajność, niezawodność i skalowalność.
    \item Magistrala i pierścień są proste, ale mają ograniczoną odporność na awarie.
    \item Gwiazda i drzewo są efektywne dla systemów hierarchicznych.
    \item Siatka zapewnia wysoką niezawodność kosztem złożoności.
    \item Hybrydowe podejście łączy różne modele, aby uzyskać optymalne rozwiązanie.
\end{itemize}
